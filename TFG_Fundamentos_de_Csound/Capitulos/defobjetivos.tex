\chapter{Definici\'on de objetivos}\label{defobjetivos}
Estos son los objetivos generales del presente TFG. Se agrega a cada uno de ellos una serie de objetivos específicos:

\begin{itemize}
\item Dar a conocer en mayor medida el lenguaje de programación Csound.
	\begin{itemize}
	\item Expresar el contenido de los capítulos en formato divulgativo.
	\item Transmitir de manera feaciente las ventajas y desventajas de Csound en un capítulo introductorio.
	\end{itemize}
\item Proporcionar una guía de aprendizaje introductorio al lenguaje Csound.
	\begin{itemize}
	\item Mostrar los contenidos en orden por dificultad, para que se pueda seguir el documento linealmente.
	\item Mostrar, al menos superficialmente, varios ámbitos del lenguaje como el ``Live Coding'', la sintaxis base del lenguaje, y otras posibilidades. 
	\item Dar una introducción práctica al IDE Cabbage.
	\end{itemize}
\item Proporcionar un documento de consulta rápida de conceptos del lenguaje Csound.
	\begin{itemize}
	\item Mostrar los contenidos de forma unitaria, proporcionando además un índice ordenado y pragmático.
	\item Proporcionar anexos con contenido relacionado como los fundamentos del sonido y otros conceptos útiles.
	\end{itemize}
\item Proporcionar ejemplos prácticos de programación usando el lenguaje Csound a modo de demostración capacitiva del lenguaje.
	\begin{itemize}
	\item Acompañar la mayoría de lecciones con al menos un ejemplo de código real.
	\item Dar referencias a otros portales de contenido de Csound que cuenten con una librería de ejemplos prácticos.
	\end{itemize}
\item Compilar una lista de referencias a portales de contenido de Csound de manera ordenada y comentada.
	\begin{itemize}
	\item Realizar una bibliografía comentada, centrada especialmente en las principales fuentes del documento sirviendo así de guía para saber por dónde seguir para extender conocimientos.
	\end{itemize}
\end{itemize}