\cdpchapter{Resumen}


El lenguaje de programaci\'on Csound est\'a principalmente destinado a la s\'intesis y producci\'on de sonido en un \'ambito musical.
Es un lenguaje de c\'odigo abierto, accesible en la plataforma GitHub y su compilador está programado en C, de ahí su nombre.
Recibe peri\'odicamente contribuciones de desarrolladores de diferentes partes del mundo y se encuentra en su versi\'on 6.14.0 a la fecha de realizaci\'on de este trabajo.
Es relativamente poco conocido al ser de uso muy específico y al estar toda su documentación dedicada a personas angloparlantes.

Al ser \'estas las circunstancias del lenguaje se ha formado a su alrededor una comunidad dedicada que se agranda con el paso del tiempo, con el fin de seguir desarrollando su tecnología y 
de ahí la justificación del principal motivo del presente trabajo de fin de grado: \bigskip

\begin{center}
\textbf{Dar a conocer los fundamentos del lenguaje Csound.}\bigskip
\end{center}

Para tal fin, se estructurará el presente trabajo a modo de guía introductoria de uso del lenguaje. Destacando las principales características de éste y priorizando la escalada progresiva de 
complejidad al decidir el orden de exposición de los conceptos, con intención de favorecer el aprendizaje a medida que se vaya usando el documento. Se priorizará también que el contenido se exponga de manera unitaria 
para favorecer el uso del presente documento como guía de consulta rápida de conceptos básicos de Csound.\bigskip

Se exponen a continuación los principales objetivos: 
\begin{itemize}
\item Dar a conocer en mayor medida el lenguaje de programación Csound.
\item Proporcionar una guía de aprendizaje introductorio al lenguaje Csound.
\item Proporcionar un documento de consulta rápida de conceptos del lenguaje Csound.
\item Proporcionar ejemplos prácticos de programación usando el lenguaje Csound a modo de demostración capacitiva del lenguaje.
\item Compilar una lista de referencias a portales de contenido de Csound de manera ordenada y comentada.
\end{itemize}

Por último destacar que tecnologías como Csound invitan al trabajo colaborativo e interdisciplinar en distintos ámbitos como son en este caso la informática y la música. Es por ello fundamental dar a conocer sus diferentes usos con el fin último de ampliar 
el desarrollo de los conocimientos tecnológicos.


