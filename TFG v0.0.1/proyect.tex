% !TEX encoding = UTF-8 Unicode
\documentclass{pclass}
\usepackage[utf8]{inputenc} % para linux y mac 
  
%DIFERENTES TIPOS DE LETRA
\usepackage{palatino}
\usepackage{times}  
\usepackage{csquotes}  


\begin{document}
\tipo{Grado}   % Grado o M\'aster
\titulopro{Fundamentos del Lenguaje Csound}
\tutor{Víctor Jesús Díaz Madrigal}
\departamento{Lenguajes y Sistemas Informáticos}
%\autores{Nombre 1}{Nombre 2}  % Dos autores
\autores{Rafael Escudero Lirio}{{\ }}   % Un autor
\dia{septiembre de 2020 (v.0.0.1)}
\titulacion{Grado en Ingeniería Informática de Computadores}

\hacerportada

	\makeatletter
\renewcommand*\l@section{\@dottedtocline{1}{0em}{2.5em}}
\renewcommand*\l@subsection{\@dottedtocline{2}{1.5em}{3.2em}}
\renewcommand*\l@subsubsection{\@dottedtocline{3}{4.3em}{3.2em}}
\makeatother

\renewcommand{\frontmatter}{\pagenumbering{Roman}}
\frontmatter
        
    \cdpchapter{Resumen}


El lenguaje de programaci\'on Csound est\'a principalmente destinado a la s\'intesis y producci\'on de sonido en un \'ambito musical.
Es un lenguaje de c\'odigo abierto, accesible en la plataforma GitHub y su compilador está programado en C, de ahí su nombre.
Recibe peri\'odicamente contribuciones de desarrolladores de diferentes partes del mundo y se encuentra en su versi\'on 6.14.0 a la fecha de realizaci\'on de este trabajo.
Es relativamente poco conocido al ser de uso muy específico y al estar toda su documentación dedicada a personas angloparlantes.

Al ser \'estas las circunstancias del lenguaje se ha formado a su alrededor una comunidad dedicada que se agranda con el paso del tiempo, con el fin de seguir desarrollando su tecnología y 
de ahí la justificación del principal motivo del presente trabajo de fin de grado: \bigskip

\begin{center}
\textbf{Dar a conocer los fundamentos del lenguaje Csound.}\bigskip
\end{center}

Para tal fin, se estructurará el presente trabajo a modo de guía introductoria de uso del lenguaje. Destacando las principales características de éste y priorizando la escalada progresiva de 
complejidad al decidir el orden de exposición de los conceptos, con intención de favorecer el aprendizaje a medida que se vaya usando el documento. Se priorizará también que el contenido se exponga de manera unitaria 
para favorecer el uso del presente documento como guía de consulta rápida de conceptos básicos de Csound.\bigskip

Se exponen a continuación los principales objetivos: 
\begin{itemize}
\item Dar a conocer en mayor medida el lenguaje de programación Csound.
\item Proporcionar una guía de aprendizaje introductorio al lenguaje Csound.
\item Proporcionar un documento de consulta rápida de conceptos del lenguaje Csound.
\item Proporcionar ejemplos prácticos de programación usando el lenguaje Csound a modo de demostración capacitiva del lenguaje.
\item Compilar una lista de referencias a portales de contenido de Csound de manera ordenada y comentada.
\end{itemize}

Por último destacar que tecnologías como Csound invitan al trabajo colaborativo e interdisciplinar en distintos ámbitos como son en este caso la informática y la música. Es por ello fundamental dar a conocer sus diferentes usos con el fin último de ampliar 
el desarrollo de los conocimientos tecnológicos.



   % \cdpchapter{Agradecimientos}

A nuestros alumnos y a nuestras alumnas.	

	\tableofcontents % Índice de contenidos
 	%\listoftables % Índice de cuadros
 	\listoffigures % Índice de figuras
 	\lstlistoflistings %Índice de códigos

\mainmatter  
    
     % !TEX root = ../proyect.tex

\chapter{Introducción al Lenguaje y Fundamentos del Sonido}\label{cap1}
\section{¿Qué es Csound?}\label{sec:intro}

Csound es un lenguaje de programación de alto nivel orientado a objetos dedicado a la síntesis, edición y producción de sonido. Su sintaxis es concreta y su compilador está codificado en lenguaje C. Entre los usos prácticos del lenguaje podemos encontrar ejemplos en la página oficial (https://csound.com/projects.html) de proyectos dedicados a la síntesis de música en directo, edición de sonido mediante efectos generados programáticamente y creación de interfaces VST o instrumentos virtuales entre otros.

Podemos referirnos a Csound como un ``Compilador de Sonido''.

\section{¿Por qué usar Csound?}\label{sec:intro}

Hay muchas buenas razones para usar Csound. Csound es software libre de código abierto con licencia LGPL, está en constante desarrollo y tiene una comunidad de desarrolladores que crece día a día,proporciona además un punto de conexión entre la disciplina informática y el ámbito de la música y el sonido. 
Es además una gran herramienta científica al facilitar la exposición y experimentación de conceptos relacionados a las ondas del sonido, con una amplia librería de funciones y objetos útiles para ello. 
En lo que se refiere a la composición musical, Csound se ha usado principalmente para generar música electrónica a lo largo de su historia aunque podemos encontrar a compositores de cualquier género músical o tipo de instrumento sacándole provecho al lenguaje. No es de extrañar vistas las tendencias actuales en el mundo de la música, donde para triunfar a nivel multitudinario es prácticamente imprescindible contar con un buen productor que sepa embellecer el sonido.
Es frecuente ver a usuarios del lenguaje interpretando su música en directo con ayuda directa de éste.
Y por último debo destacar que Csound es compatible con los principales sistemas operativos del mercado, desde Windows a iOS pasándo por distrbuciones Linux. Y como veremos más adelante en esta guía, sus funciones pueden llamarse desde el código de otros lenguajes como Python, java o C.
\pagebreak

\section{Breve historia de Csound}\label{sec:intro}

De manera resumida, Csound fue desarrollado en un principio por Barry Vercoe (véase la figura 1.1) en 1985 en el MIT\footnote{Massachusetts Institute of Technology} Media Lab. Y desde la década de los años 1990, una amplia variedad de desarrolladores ha colaborado a su código abierto, aportando además documentación, ejemplos y artículos sobre el lenguaje.

\figura{0.5}{img/1.1-Barry-Vercoe}{Barry Vercoe}{fig:barry}{}

Para hablar de los verdaderos orígenes de Csound debemos remontarnos a la década de los años 1970, a los orígenes de la producción informática de sonido. 
Matt Mathews creó MUSIC, el primer lenguaje informático para la generación de ondas de audio digital. En éste se basarían sus posteriores iteraciones: Music1, Music2, Music3, Music4, Music4B... Hasta llegar a Music11, desarrollado por el mentado Barry Vercoe, y del cual Csound es sucesor directo.
Posteriormente el desarrollo del lenguaje ha continuado gracias a su comunidad con John Fitch de la University of Bath a la cabeza y como dueño del repositorio de código abierto en la plataforma GitHub.
En la actualidad se realizan periódicamente las ICSC\footnote{International Csound Conference}, conferencias dedicadas a Csound a nivel internacional de manera periódica, siendo la última fecha de realización el 27 de septiembre de 2019 en la actualidad del presente documento.

\section{Las características del lenguaje}\label{sec:intro}

A continuación se muestra una lista de las características principales y técnicas del lenguaje:
\begin{itemize}
 \item Usado para la síntesis, edición y análisis de sonido y música.
 \item Lenguaje de código abierto.
 \item Programación funcional.
 \item Licencia de distribución LGPL.
 \item Orientado a objetos.
 \item Compilador programado en lenguaje C.
 \item 30 años de desarrollo.
 \item Compatibilidad con otros lenguajes y retrocompatibilidad de versiones.
\end{itemize}



\section{Alternativas a Csound}\label{sec:intro}

La Normativa académica de los Trabajos fin de Grado indica:

\emph{Como norma general, el TfG deberá estar escrito y ser expuesto oralmente en castellano.
Podrá también estar escrito y ser expuesto en inglés, previa solicitud.}

En cualquier caso, la memoria debe respetar los usos y costumbres del idioma en que sea escrita.
Debe prestarse especial atenci\'on al guionado de las palabras, debido a que muchas aplicaciones inform\'aticas
usan el propio del ingl\'es y no el del castellano.

En cuanto a la portada, se debe utilizar la portada oficial de la ETSII. Un ejemplo de la misma puede encontrarse al 
principio (como portada) de este documento.


 

     % !TEX root = ../proyect.tex

\chapter{La Sintaxis del Lenguaje}

\section{Hello Csound!}\label{sec:hello}

Empecemos por mostrar cómo sería el clásico \textsl{Hello World!} en Csound. Así tendremos un código básico al que hacer referencia a los largo de este capítulo\footnote{Durante el resto del presente capítulo se hará referencia a la figura de código anterior siempre que se hable de líneas concretas de código.}:
\codigofuente{TeX}{Hello World! en Csound}{codigo/helloWorld}

\section{A tener en cuenta}\label{sec:cuenta}

Estas son algunas consideraciones básicas del lenguaje que debemos tener presentes:
\begin{itemize}
 \item Es sensible a mayúsculas/minúsculas.
 \item Usa especificadores de formato al imprimir variables.
 \item Para realizar comentarios en el código usaremos ; al inicio de la línea o usar /* y */ para comentar varias líneas.
 \item Partes del código divididas en etiquetas XML.
\end{itemize}

\section{División por etiquetas}\label{sec:etq}

Csound divide la estructura de su código con etiquetas XML, empezaremos por hablar de las más básicas: \textless CsInstruments\textgreater y \textless CsScore\textgreater. Hasta llegar a etiquetas más exclusivas como \textless Cabbage\textgreater.

\begin{itemize}
 \item \textbf{Etiqueta \textless CsInstruments\textgreater}: En esta etiqueta se incluirán las definiciones de los instrumentos que crearemos. Un poco más adelante trataremos de entender qué es un instrumento en Csound. En nuestro ejemplo la etiqueta \textless CsInstruments\textgreater abarca desde la línea 4 a la línea 12, y podemos observar la definición de un instrumentos entre las líneas 6 y 10.
 \item \textbf{Etiqueta \textless CsScore\textgreater}: Aquí haremos uso práctico de nuestros instrumentos, les diremos cómo ejecutarse y durante cuánto tiempo. En el código del que disponemos, la etiqueta \textless CsScore\textgreater abarca desde la línea 13 a la 15 y tenemos un ejemplo sencillo de ejecución en la línea 14 al que volveremos más adelante.
 \item \textbf{Etiqueta \textless CsOptions\textgreater}: Se inclurán aquí las especificaciones técnicas para interactuar con hardware u otros dispositivos.
 \item \textbf{Etiqueta \textless CsoundSynthesizer\textgreater}: Todo el código, incluidas las etiquetas mencionadas anteriormente, debe estar incluido en \textless CsoundSynthesizer\textgreater. Es la forma que tiene el compilador de saber dónde empieza y dónde acaba el código Csound.
 \item \textbf{Etiqueta especial \textless Cabbage\textgreater}: Es la única etiqueta que no debe estar dentro de \textless CsoundSynthesizer\textgreater puesto que es exlusiva de Cabbage, IDE que usaremos principalmente en este documento. En esta etiqueta incluiremos el código referente a las opciones de personalización de interfaz gráfica de nuestro programa, hablaremos más de ella en secciones referentes al uso del IDE Cabbage.
\end{itemize}

\section{Palabras reservadas}\label{sec:reservadas} 



\section{Plantillas}\label{sec:plantillas} 

Se encuentra disponible una plantilla LaTex en la secci\'on de documentos de la plataforma
de la aplicaci\'on de TfG de la ETSII 
\url{https://tfc.eii.us.es/TfG/}. Esa plantilla ha sido utilizada para preparar este documento.
Se espera en breve disponer de plantillas de ejemplo para las aplicaciones OpenOffice Writer y Office Word.

Nota: Las plantillas se proporcionan como ejemplo, las condiciones obligatorias son las que constan en este procedimiento.

\chapter{Ejemplos}\label{cap2} 
					
\section{Manejo de la bibliograf{\'\i}a}
En esta secci\'on mostramos brevemente ejemplos de referencia a la bibliograf{\'\i}a 
citando un libro~\cite{desousa}, un art{\'\i}culo~\cite{guiatitlesec} 
y una p\'agina web~\cite{informatica}. 

Se recuerda que son campos obligatorios
en todos los {\'\i}tems de la bibliograf{\'\i}a: 
autor(es), t{\'\i}tulo del libro o art{\'\i}culo 
% editorial (en su caso, tambi\'en revista) 
y a{\~n}o de publicaci\'on.
En el caso de p\'aginas web, 
es obligatoria la fecha de la \'ultima consulta. 
En general, la bibliograf{\'\i}a debe ayudar al lector a encontrar f\'acilmente los {\'\i}tems citados.

\section{C\'odigo}

En general se debe evitar incluir c\'odigo o seudoc\'odigo en la memoria. Si fuese preciso, se destacar\'a de forma
que sea f\'acilmente identificable y se indexar\'an los trozos de c\'odigo incluidos. Un ejemplo puede verse
a continuaci\'on.

\codigofuente{TeX}{Código de ejemplo en LaTeX}{codigo/macrotabla}

\section{Im\'agenes}

Este es un ejemplo de inclusi\'on de figura en el texto (v\'ease la figura~\ref{fig:ada}). 

\figura{0.5}{img/AdaLovelace}{Ada Lovelace}{fig:ada}{}

Figuras y cuadros se colocar\'an preferentemente tras el p\'arrafo en el que
son llamados por primera vez. Si no cupieran, se colocar\'an (en orden de preferencia):
\begin{itemize}
 \item Al final de la p\'agina en que se llaman
 \item Al principio de la siguiente p\'agina
 \item Al final del cap{\'\i}tulo
\end{itemize}
siempre respetando el orden de aparici\'on en el texto.

\section{Cuadros (mal llamados Tablas)}

Este es un ejemplo de inclusi\'on de cuadro en el texto. V\'ease el cuadro~\ref{tab:prueba}

\cuadro{|r||c|c|}{Cuadro de prueba}{tab:prueba}{elemento & elemento & elemento \\ elemento & elemento & elemento \\}
     % !TEX root = ../proyect.tex

\chapter{Profundizando en los conceptos básicos}

Profundicemos en algunos conceptos que crean normalmente confusión entre los usuarios de Csound.

\section{Las diferencias entre variables i-rate y k-rate}

Es común confundir el uso de variables \textbf{i-rate} con el de variables \textbf{k-rate}. Vamos a resolver algunas dudas y a exponer ejemplos de casos en los que usar normalmente cada tipo.

En primer lugar  debemos entender que, como en cualquier lenguaje al uso, las variables en Csound se inicializan al comenzar la ejecuación. La diferencia radical que podemos encontrar a partir de este momento entre variables \textbf{i-rate} y \textbf{k-rate} es que las \textbf{i-rate} van a quedarse con este valor de inicialización. Esto es fácil de entender pero, ¿Qué pasa entonces con las variables \textbf{k-rate}?\bigskip

La duración de un \textbf{k-cycle}, es decir, el tiempo (que puede medirse en cantidad de samples) que pasa desde la útima vez que se refrescaron los valores de las variables \textbf{k-rate} hasta la siguiente vez que se refrescan.

Este valor, \textbf{k-cycle}, es dinámico. Depende de la cantidad de variables tipo k de nuestro código, del valor que le demos a la palabra reservada \textbf{ksmps} y del sample rate seleccionado. Veámoslo con un ejemplos:


\figura{1}{img/3.1-kCycles}{La duración de los K-Cycles}{fig:kCycles}{}

Para el gráfico anterior hemos seleccionado un sr (sample rate) de 44100 y un Block Size (cuyo equivalente en Csound sería la palabra reservada \textbf{ksmps} que mencionamos antes) con valor 10.

Sabemos que el tiempo que tarda en realizarse una muestra o sample es un segundo dividido por el número total de tomas, es decir:\( 1/44100\) que da 0.0000227s.
Y sabemos también que el \textbf{k-cycle} dura 10 samples (también podemos llamarlos ticks), es decir: \( 0.0000227*10= 0.000227s \).

Por lo tanto, en un código con la configuración anterior y un valor de \textbf{ksmps = 10} podremos decir que:

\begin{itemize}
 \item Las variables \textbf{i-rate} determinarán su valor en la inicialización y lo mantendrán durante el tiempo total de ejecución del código.
 \item Las variables \textbf{k-rate} refrescarán su valor cada 10 samples o ticks (que podemos llamar bloque de control), en este caso, se refrescarán cada 0.000227 segundos.
 \item Las variables \textbf{a-rate} refrescarán su valor un vez por cada sample, siendo el ratio de refresco determinado en su totalidad por la frecuencia escogida como sr. En este ejemplo renuevan su valor cada 0.0000227 segundos.
\end{itemize}

Tener un conocimiento suficiente acerca del funcionamiento de las variables \textbf{k-rate} puede hacer que nuestro código acabe siendo mucho más eficiente y optimizado pero como nota general puede bastarnos con recordar que: Las variables \textbf{i-rate} deben usarse cuando sepamos que algo debe ser hecho una única vez y de manera puntual, y las variables \textbf{k-rate} deben usarse cuando necesitemos que algo se haga continuamente pero sepamos que ese algo no necesita ser hecho cada vez que se realiza un sample.

\section{Las f-variables,  w-variables y S-variables}

Existen dos tipos de variables en csound que son algo más especiales que las vistas hasta ahora:

\begin{itemize}
 \item \textbf{Las f-variables}: Son variables usadas por algunos opcodes (los que empiezan por \textbf{pvs}), y se usan principalmente para la realización de \textsl{Transformadas rápidas de Fourier}. Su ratio de refresco es el mismo que para las variables \textbf{k-rate} pero su valor depende de algunos parámetros  de las transformadas que hemos mencionado.
 \item \textbf{Las w-variables}: Podemos encontrar el uso de estas variables en algunos opcodes antiguos aunque su uso es ya prácticamente por razones hereditarios por lo que no profundizaremos en ellas.
 \item \textbf{Las S-variables}: Son variables de tipo \textbf{String}, serán necesarias para usar algunos opcodes cuyo resultado sea una variable de este tipo.
\end{itemize}

\section{El ámbito global y local de las variables}

Las variables contenidas en el código de un instrumento son generalmente de ámbito local, es decir, podría crear una variable con el mismo nombre dentro de todos mis instrumento sin que exista ningún tipo de conflicto.

Las variables globales sin embargo deben ser únicas cada vez que escribamos un valor sobre ellas, su valor cambiará para cualquier futura lectura. Para hacer que una variable sea de tipo global debemos hacer que \textbf{g} sea la primera letra del nombre de esa variable. Algunos ejemplo de nombre de variable global serían: \textbf{gaGlobal}, \textbf{giConstante} o \textbf{gkResultado}.

Una alternativa al uso de las variables globales es el uso del opcode \textbf{chnget} para realizar conexiones de canal entre variables. Es el método usado en la etiqueta \textless Cabbage\textgreater para relacionar los widgets con el resto del código sin hacer uso de variables globales.

\section{Las estructuras de control}

En Csound como en la mayoría de lenguajes existen la estructuras de control: Sentencias \textbf{if-else}, bucles \textbf{while/until} y los llamados \textbf{timouts}. Vamos a explicarlas centrándonos en la peculiaridades de uso respecto al lenguaje.

\subsection{Sentencias if-else}

La forma más común de este tipo de sentencia en Csound es \textbf{If - then - [elseif - then -] else}:

\codigofuente{TeX}{Sintaxis base de la sentencia if-else}{codigo/ifejemplo}

Lo único a destacar sería que la palabra \textbf{then} debe estar en la misma línea de código que la palabra \textbf{if}, pero no ahondaremos más en este tipo de sentencia al tratarse de nociones básicas de la programación.

Csound permite también la sintaxis de lenguaje descriptivo \textbf{(a v b ? x : y)}: De ser verdadera la condición \textbf{a}, el valor devuelto es \textbf{x}. De ser falsa (y por tanto ser cierta \textbf{b}) el valor devuelto es \textbf{y}. Un ejemplo práctico de uso sería: \textbf{kRes = (kVar \textless 1 ? 0 : 1);}. Si kVar es menor que uno se devuelve 0, se no ser así se devuelve 1.

\subsection{Bucles While/Until}

\codigofuente{TeX}{Sintaxis base de los bucles while-until}{codigo/whileejemplo}

Estos bucles funcionan de forma análoga, la única diferencia entre ellos es que el bucle \textbf{while} se seguirá ejecutando siempre y cuando la condicion sea verdadera y el bucle \textbf{until} se seguirá ejecutando siempre y cuando la condición sea falsa.

\subsection{El timout}

El \textbf{timout} es un opcode para generar bucles de una duración determinada. 

\codigofuente{TeX}{Sintaxis base del timout}{codigo/timoutSyntax} 

En primer lugar \textbf{first\_ label} y \textbf{second\_ label} son etiquetas de referencia a las que podemos saltar desde otras partes del código. El opcode \textbf{(timout    istart, idur, second\_ label)} tiene tres parámetros de entrada: \textbf{istart} el instante de inicio, \textbf{idur} la duración de timout y \textbf{second\_ label} el nombre de la etiqueta de la parte del código a la que queremos saltar, en este caso por lógica la segunda.El segundo opcode necesario para el funcionamiento es \textbf{(reinit	first\_ label)} que da la orden directa de saltar a la parte del código referida por \textbf{first\_ label}.

Entendamos el uso de \textbf{timout} con un ejemplo práctico:

\codigofuente{TeX}{Ejemplo real de uso del timout}{codigo/timoutReal}

Hemos definido una variable \textbf{idur} a la que damos un valor aleatorio entre (0,5 y 3) mediante el opcode \textbf{random}. Acto seguido usamos el opcode \textbf{timout} para saltar desde el instante \textbf{0}, durante ese \textbf{valor aleatorio de segundos}, a la etiqueta \textbf{play} donde se ejecuta el opcode \textbf{poscil} que genera una onda de sonido. Una vez acabado ese periodo de tiempo se ejecuta el opcode \textbf{reinit} que nos hace saltar a la etiqueta \textbf{loop} y vuelta a empezar.

\section{Los Arrays de datos}

Al igual que en cualquier lenguaje con cierto nivel de complejidad, en Csound existen los \textbf{Arrays} o vectores de datos. Veamos las peculiaridades del lenguaje en el uso de este tipo de estructuras.

\subsection{Propiedades de los Arrays}

En Csound podemos pensar en cinco propiedades características al definir un Array:

\begin{itemize}
 \item \textbf{Dimensiones}: Los elementos de un array pueden leerse mediante la sintaxis \textbf{kArr[i]} siendo \textbf{i} la posición del elemento al que queremos acceder en un array unidimensional. De igual manera podemos acceder a los elementos de un array de dos dimensiones con la sintaxis \textbf{kArr[i][j]}. De forma análoga para arrays tridimensionales.
 \item \textbf{i- o k-Rate}: Los arrays son variables y como tal pueden definirse como variable i-rate o k-rate.
 \item \textbf{Local o Global}: De la misma manera podemos hacer de nuestro array una variabe global añadiendo la letra \textbf{g} al principio del nombre.
 \item \textbf{Arrays de Strings}: Los arrays de Csound pueden contener variables String además de número por lo que podemos clasificarlos también de esta manera.
 \item \textbf{Arrays de señales digitales}: Por último podemos conectar canales y salidas de opcodes a las posiciones de un array para facilitar el trabajo con las señales de audio.
\end{itemize}
\pagebreak

\subsection{Opcodes útiles}

Estos son algunos de los opcodes que podemos usar al trabajar con arrays:

\begin{itemize}
 \item \textbf{init}: El opcode que usamos para inicializar un array, para su sintexis únicamente necesitamos aportar el tamaño de cada una de las dimensiones del array que estamos definiendo. Por ejemplo: \textbf{kArr[]   init 5} para un array unidimensional de longitud 5.
 \item \textbf{fillarray}: Para añadir una serie de valores a nuestro array. Veamos un ejemplo: \textbf{kArr[] fillarray 1, 2, 3, 4, 5} que añade los valores 1, 2, 3, 4 y 5 al array.
 \item \textbf{genarray}: Genera un array al que se le añaden los valores comprendidos entre los valores de entrada del opcode. Un ejemplo de uso: \textbf{kArr[] genarray   1, 5} que crea un array al que se le añaden los valores 1, 2, 3, 4 y 5.
 \item \textbf{lenarray}: Devuelve el tamaño actual de un array. \textbf{kTam  lenarray  kArr} devolvería el tamaño de kArr. 
 \item \textbf{slicearray}: Nos sirve para generar un subarray desde un índice inicial hasta un índice final del array original introducido como parámetro de entrada en el opcode. Su sintaxis base es: \textbf{kSlice[] slicearray kArr, iStart, iEnd}. Y un ejemplo de uso sería \textbf{kSub[]  slicearray kArr, 0, 2} que para el array \textbf{kArr=[4,3,2,1,0]} devolvería como resultado \textbf{kSub=[4,3,2]}
 \item \textbf{minarray}: Este opcode devuelve el valor más pequeño de todo el array y de manera opcional su índice si especificamos una variable para guardarla. Su sintaxis base  es: \textbf{kMin [,kMinIndx] minarray kArr}.
 \item \textbf{maxarray}: Este opcode devuelve el mayor valor de todo el array y de manera opcional su índice si especificamos una variable para guardarla. Su sintaxis base  es: \textbf{kMin [,kMinIndx] minarray kArr}.
 \item \textbf{sumarray}: Devuelve la suma de todos los valores del array numérico. Su sintaxis es \textbf{kSum sumarray kArr} y siendo \textbf{kArr=[1,1,1,1]} resultaría \textbf{kSum = 3}.
 \item \textbf{scalearray}: Escala los valores de un array en referencia a un valor mínimo y a un valor máximo. Veamos un ejemplo de uso: 
  
\codigofuente{TeX}{Uso del scalearray}{codigo/scaleArray}  

Tenemos el array \textbf{kArr=[1,3,9,5,6]}, donde el valor más pequeño es 1 y el valor más grande es 9. Cuando hacemos uso del opcode \textbf{(scalearray kArr 1,3)} estaremos escalando con las siguientes referencias relativas: \textbf{1 $\leftrightarrow$ 1 (kmin)} y \textbf{9 $\leftrightarrow$ 3 (kmax)}. De la misma manera el resto de valores de \textbf{kArr} quedarán en referencia también a kmin y kmax. El resultado para \textbf{(scalearray kArr 1,3)} con \textbf{kArr=[1,3,9,5,6]} sería  \textbf{kArr=[1, 1.5, 3, 2, 2.25]}.

 \item \textbf{maparray}: Aplica un opcode en formato función a cada uno de los valores del array. Su sintaxis base es: \textbf{(kRes  maparray kArr, ``fun")}. Siendo \textbf{fun} la función que queremos usar de entre las siguientes opciones: \textbf{abs, ceil, exp, floor, frac, int, log, log10, round, sqrt}. Por ejemplo, la operación \textbf{(kRes  maparray kArr, sqrt)} aplica la función \textbf{sqrt()} a cada elemento del array \textbf{kArr} y almacena los resultados en \textbf{kRes}.
  
\end{itemize}

\subsection{Operaciones con arrays}

Se pueden usar los cuatro operadores básicos (\textbf{+, -, *, /}) para sumar, restar, multiplicar y dividir arrays. Si el operador se aplica entre el array y un número, la operación se realiza a todos los elementos del array, por lo que guardar los resultados en un nuevo array es una buena práctica. Por ejemplo si tenemos el array \textbf{kArr=[1,1,1]} y realizamos la operación \textbf{kSuma = Karr + 1} obtendremos \textbf{kSuma=[2,2,2]}.

Por otra parte, si realizamos la operación entre dos arrays del mismo tamaño, el cálculo tendrá en cuenta cada pareja de valores de la misma posición de cada array. Si tenemos el array \textbf{kArr1=[10,10,10]} y el array \textbf{kArr2=[1,2,3]} y realizamos la operación \textbf{kArr3 = kArr1 + kArr2} obtenemos \textbf{kArr3=[11,12,13]}.

\section{Funciones de entrada/salida}

Muchos de los opcodes de Csound pueden expresarse mediante la sintaxis funcional que se usa en tantos otros lenguajes: \textbf{fun(arg1, agr2...)}. Hablemos de algunos de los opcodes que más se usan en su formato funcional.

\begin{itemize}
 \item \textbf{abs}: Devuelve el valor absoluto del parámetro de entrada \textbf{arg}. 
 
 Sintaxis: \textbf{abs(arg)}
 
 Ejemplo: \textbf{abs(-2) = 2}
 \item \textbf{ceil}: Devuelve el menor número entero posible que sea mayor que \textbf{arg}.
 
 Sintaxis: \textbf{ceil(arg)}
 
 Ejemplo: \textbf{ceil(0.999) = 1}
 \item \textbf{exp}: Devuelve el número \textbf{e (2,718...)} elevado a la potencia de \textbf{arg}.
 
 Sintaxis: \textbf{exp(arg)}
 
 Ejemplo: \textbf{exp(2) = 7.389}
 \item \textbf{frac}: Devuelve la parte fraccionaria de \textbf{arg}.
 
 Sintaxis: \textbf{frac(arg)}
 
 Ejemplo: \textbf{frac(2.1234) = 0.1234}
 \item \textbf{int}: Devuelve la parte entera de \textbf{arg}.
 
 Sintaxis: \textbf{int(arg)}
 
 Ejemplo: \textbf{int(2.1234) = 2}
 \item \textbf{log}: Devuelve el logaritmo natural de \textbf{arg}.
 
 Sintaxis: \textbf{log(arg)}
 
 Ejemplo: \textbf{log(8) = 2.079}
 \item \textbf{sqrt}: Devuelve la raíz cuadrada de \textbf{arg}.
 Sintaxis: \textbf{sqrt(arg)}
 
 Ejemplo: \textbf{sqrt(25) = 5}
\end{itemize}

\section{Creando un opcode}

\section{Las macros}

     % !TEX root = ../proyect.tex

\chapter{Conceptos Avanzados}

\section{El opcode ftgen}

el opcode \textbf{ftgen} es muy recurrente. Nos sirve para generar una tabla de scores de entre nuestros instrumentos.

Su sintaxis base es:

\codigofuente{TeX}{Sintaxis base de ftgen}{codigo/ftgenSyntax}

Siendo \textbf{gir} una tabla de al menos 100 posiciones, \textbf{ifn} el número de la tabla. El resto de valores se corresponden con los argumentos \textbf{p2}, \textbf{p3}, \textbf{p4} y \textbf{p5} del \textbf{f Statement} que explicaremos a continuación.

\subsection{El f statement}

Coloca valores en una tabla de scores para usar nuestros instrumentos. Está implícita en \textbf{ftgen}.

Esta es la sintaxis base:

\codigofuente{TeX}{Sintaxis base de ftgen}{codigo/fStateSyntax}

Siendo \textbf{p1} el número de tabla, \textbf{p2} tiempo de activación para generar datos, \textbf{p3} tamaño de la tabla, \textbf{p4} nombre de la rutina \textbf{GEN} de generación de datos y \textbf{p5 ... PMAX} que dependerán del \textbf{GEN} que estemos usando.

Veamos un ejemplo en el que rellenamos una tabla (f1) de tamaño 5 con ceros: \textbf{( f1 0 5 2 0\ )}

\section{Delay y Feedback}

Empecemos por explicar en qué consisten estos efectos. 

\subsection{El Delay}
Podemos definir el \textbf{Delay} como un buffer de repetición de nuestra onda que va poniendo en cola los sonidos que van entrando y según las características de nuestro \textbf{Delay} los saca de una manera determinada, conformando así algunos de los efectos más conocidos en la producción musical. 

El uso más básico del \textbf{Delay} es el clásico efecto ``eco'' pero también puede servirnos para implementar el Chorus, Flanging, Cambio tonal y otros filtros de onda.

Respecto a Csound existen varios opcodes dedicados al \textbf{Delay}, como \textbf{delayr} y \textbf{delayw}.

Empecemos por la diferencia en los nombres de estos opcodes que se usan en conjunto puesto que \textbf{delayr} es de lectura (read) y \textbf{delayw} es de escritura (write) del buffer de delay. 

Veamos un ejemplo básico de uso:

\codigofuente{TeX}{Sintaxis delayr y delayw}{codigo/delaySyntax}

Donde \textbf{aSignIn} es la señal de entrada a la que queremos aplicar el delay y \textbf{aSignOut} es la señal de salida a la que ya se ha aplicado el efecto.
El '1' que aparece como argumento es la cantidad de segundos que mide el buffer, en este caso un segundo y el hecho de definir aquí la longitud del buffer implica que tengamos que definir primero la variable de salida antes y después la variable de entrada, lo cual es en principio un poco confuso.

\subsection{El Feedback}

El delay tal y como lo hemos definido anteriormente acaba pareciendo una simple única repetición atenuada del sonido de entrada. Para encontrar el clásico sonido al que en el mundo de la edición musical nos referimos como delay debemos conocer el concepto de \textbf{feedback}.

El \textbf{feedback} será un valor residual de la señal del buffer que usaremos como parámetro de entrada del delay. Lo normal es que este valor decrezca con cada paso por el buffer de manera que el sonido acabe pareciendo cada vez más distante hasta apagarse por completo y no se produzca un bucle infinito.

Respecto a Csound podremos implementar este concepto de manera sencilla con una variable.

\subsection{Ejemplo del efecto Delay}

Veamos a continuación el código de un efecto delay con feedback y analicémoslo con el objetivo de comprenderlo mejor:

En primer lugar podemos observar tres opcodes que no habíamos visto hasta el momento. \textbf{loopseg} en la línea 16, \textbf{randomh} en la línea 17 y \textbf{interp} en la línea 18. 

Expliquemos brevemente sus funcionamientos:

\begin{itemize}
 \item \textbf{loopseg}: Genera una señal de control que puede usarse como envelope o envoltura. Tantovariable de salida como variables de entradas serán de tipo k-rate.
 \item \textbf{randomh}: Genera valores aleatorios al igual que el opcode \textbf{random} pero mantiene estos valores guardados durante un periodo determinado de tiempo.
 \item \textbf{interp}: Sirve para convertir una variable de tipo k-rate a una variable de tipo a-rate.
\end{itemize}

Una vez sabido el funcionamiento básico de cada opcode presente veamos paso por paso lo que  se hace en el ejemplo:

\begin{itemize}
 \item \textbf{Línea 13}: Hacemos uso de \textbf{ftgen} para tener preparada nuestra onda senuidad.
 \item \textbf{Líneas 15 a 19}: Definimos las primeras líneas de nuestro instrumento. El uso de estas líneas es el de acabar generando \textbf{aSig}, es decir, nuestra onda de entrada a la que aplicaremos el delay.
 \item \textbf{Línea 21}: Definimos \textbf{iFback} que tal y como explicamos en secciones anteriores hará las veces de valor de feedback, en este caso 0.7 (de ser un valor menor nuestro delay se apagaría antes).
 \item \textbf{Líneas 22 a 26}: Hacemos uso de \textbf{delayr} y \textbf{delayw}. Si prestamos especial atención a la línea 23 veremos que nuestro valor de entrada, \textbf{(aSig(aBufOut*iFdback))}, es efectivamente una operación en la que participan nuestra señal, su propio valor de salida (siguiente en el buffer) y el valor de feedback que acabará determinando el tiempo restante del delay.
 
 Como vemos también en la línea 25, el sonido que nuestro instrumento producirá será una mezcla de la onda de salida con el valor de buffer.
\end{itemize}

\codigofuente{TeX}{Ejemplo completo del efecto Delay}{codigo/delayCompleto}

Se invita al lector a que experimente con diversos valores en las variables mencionadas para comprender en mayor medida el concepto de delay y que posteriormente visite las referencias del documento en caso querer profundizar.

\section{Modulación de frecuencia}
http://write.flossmanuals.net/csound/d-frequency-modulation/
\section{AM RM WAVESHAPING}
http://write.flossmanuals.net/csound/f-am-rm-waveshaping/
\section{AMPLITUDE AND RING MODULATION}
http://write.flossmanuals.net/csound/c-amplitude-and-ring-modulation/
\section{Filtros de onda}
http://write.flossmanuals.net/csound/c-filters/
\section{Reverberación}
http://write.flossmanuals.net/csound/e-reverberation/

\subsection{sub}


 


     % !TEX root = ../proyect.tex

\chapter{Cabbage: Guía de uso}

Cabbage es un IDE para el lenguaje Csound. Es de código abierto y está desarrollado por Rory Walsh.

Será el principal IDE que usaremos a lo largo de este documento puesto que además de contar con todas las comodidades necesarias para el funcionamiento del lenguaje, aporta además funcionalidades para crear interfaces gráficas para nuestro software de manera muy simple pero vistosa.

Se presenta el siguiente capítulo con la intención de dar una guía básica referencial de funcionamiento a la que acudir en caso de ser necesario durante el curso del resto de contenidos.

\section{Instalación de Cabbage}\label{sec:CabbageInst}

Cabbage puede instalarse en sistemas Windows, OSX y Linux. Posee incluso un instalador en versión beta para sistemas Android.

Pasos para la instalación en Windows y OSX:

\begin{itemize}
 \item Acudir a la página \url{https://cabbageaudio.com/download/} donde encontraremos los enlaces de descarga.
 \item Seleccionar la versión adecuada para nuestro sistema, en este caso Windows u OSX.
 
 \figura{0.6}{img/C.1-InstaVersion}{Versiones disponibles}{fig:InstCabbage}{}
  
 \item Ejecutamos el archivo descargado y seguimos los pasos de instalación. Para los instaladores de Windows y OSX se incluye una instalación automática del lenguaje Csound en nuestro sistema por lo que una vez instalado Cabbage todo estará listo para usar.
 \figura{0.6}{img/C.2-InstaPasos}{Pasos de la instalación}{fig:InstPasos}{}
 \item Por último podemos ejecutar el acceso directo que se instala automáticamente en nuestro escritorio y empezar a usar Csound. Como se observa, Csound tiene una instalación muy fácil en estos sistemas.
\end{itemize}

\section{Opciones del IDE}\label{sec:OpcionesCabbage}

La extensión de los archivos de código Csound es \textbf{.csd}. Por supuesto, Cabbage puede abrir y ejecutar estos archivos además de que podemos encontrar una enorme librería de ejemplos que trae la instalación del IDE por defecto:

  \figura{0.6}{img/C.3-CabbageEjemplos}{Ejemplos disponibles en el IDE}{fig:EjemplosCabbage}{}

\subsection{Creando un nuevo archivo}

Para crear un nuevo archivo .csd basta con clicar el icono correspondiente en la barra superior(el primero empezando por la izquierda) o clicando en \textbf{File\textgreater New Csound File}. Aparecerá una ventana con cuatro posibles opciones:

\figura{0.6}{img/C.4-CabbageNewFile}{Sintetizador, Efecto, Archivo Csound y VCV Rack}{fig:NewFileCabbage}{}

Se trata de cuatro plantillas que nos aporta Cabbage para facilitarnos el desarrollo de nuevos instrumentos, hablemos de cada una de estas opciones:

\begin{itemize}
 \item \textbf{Sintetizador}: Aporta el código fundamental de un teclado funcional sin efectos, del cual podemos partir como base para crear nuestros sintetizadores y añadir la serie de modificadores que deseemos.
 \item \textbf{Efecto}: Aporta el código de un efecto básico de ganancia. En base a este código podemos crear el efecto que deseemos para poder modificar a gusto las ondas de sonido generadas por nuestros instrumentos.
 \item \textbf{Archivo Csound}: Genera un archivo .csd completamente vacío, es la opción que escogeríamos si no tenemos predilección por las demás o si nuestro objetivo es crear un software no totalmente convencional a lo que suele verse en Csound.
 \item \textbf{VCV Rack}: Esta es la plantilla más novedosa hasta la fecha en Cabbage. Nos da facilidades para exportar nuestro código como módulos \textbf{VCV Rack} y al usarla genera el código base de un efecto de ganancia modular listo para ser exportado y usado en softwares de estación de trabajo digital (EAD) o (DAW) por sus siglas en inglés.
\end{itemize}


\section{La etiqueta \textless Cabbage\textgreater}\label{sec:CabbageInst}

A diferencia del resto de etiquetas, la etiqueta \textless Cabbage\textgreater es exclusiva al IDE y proporciona funcionalidades para el diseño de la interfaz de usuario. Veamos algunos de sus usos:

\subsection{Los Widgets}

Llamaremos a los diferentes elementos de interfaz gráfica que aporta Cabbage, widgets. Podemos dividirlos en dos tipos: interactivos (botones, sliders, barras de selección, etc...) y no interactivos (imágenes, indicadores, etc...).

Empecemos con un ejemplo de uso de un slider para comprender la sintaxis:

\codigofuente{TeX}{Ejemplo básico de un widget}{codigo/CabbageSyntax}

En nuestra figura vemos que para hacer uso de un widget empezamos por escribir su nombre identificativo de tipo, en este caso \textbf{rslider}. Más tarde podemos definir una serie de identificadores para personalizar nuetro widget. Para especificar la posición de nuestro widget y su tamaño usaremos \textbf{bounds(x, y, width, height)}. En el caso de nuestro ejemplo estamos posicionando nuestro slider en las coordenadas XY (10,10) y le estamos dando un tamaño de 100*100 píxeles.

Podemos usar también el identificador \textbf{range(min, max, value, skew, incr)}. Sus valores min y max marcan el mínimo y máximo valor del slider en cuestión, el resto de parámetros son opcionales. \textbf{value} indica el valor inicial del slider. \textbf{skew} puede usarse para determinar la salida de datos del slider de forma no lineal, su valor predeterminado es 1 pero al darle un valor por ejemplo de 0.5 conseguiríamos una salida de datos exponencial. \textbf{incr} determina el tamaño de los pasos incrementales que da el slider a usarlo, por ejemplo con un valor 0.4, de un valor cualquiera del slider a sus adyacentes habría necesariamente una distacia en valor a 0.4.

En nuestro ejemplo hemos creado un slider cuyo rango de valores va del 0 al 1 y cuyo valor inicial es 0.5.

\section{Algunos Widgets útiles}
Veamos algunos de los widgets más usados en Cabbage:
\subsection{Form}
El widget \textbf{Form} nos sirve para crear la ventana de nuestra interfaz de usuario. Sobre este widget colocaremos el resto de elementos de nuestra interfaz.
\figura{0.4}{img/C.6-formExample}{Widget: Form}{fig:NewFileCabbage}{}

Este es un ejemplo básico de \textbf{Form}:
\codigofuente{TeX}{Ejemplo de widget: Form}{codigo/FormEx}
Siendo \textbf{size()} el tamaño de la ventana, \textbf{caption()} el String que muestra la cabecera de la ventana, \textbf{pluginID()} el identificador del widget y \textbf{colour()} su color en formato rgb.

\pagebreak
\subsection{Check Box}
El widget \textbf{CheckBox} nos sirve para marcar una casilla que por ejemplo active y desactive la generación de una de las ondas de salida de nuestros instrumentos.
\figura{0.4}{img/C.7-CheckBox}{Widget: CheckBox}{fig:NewFileCabbage}{}

Este es un ejemplo básico de \textbf{CheckBox}:
\codigofuente{TeX}{Ejemplo de widget: CheckBox}{codigo/CheckBoxEx}
Siendo \textbf{bounds()} la posición y tamaño del widget en la ventana, \textbf{channel()} el identificador que vinculará el \textbf{checkBox} con una variable de nuestro código y \textbf{text()} El texto que aparece junto al \textbf{CheckBox} mientras esté y no esté pulsado.

\subsection{Button}
El widget \textbf{Button} crea un botón al que podemos dar cualquier tipo de uso como por ejemplo el de inicio de grabación o activación de algún instrumento desde un instante determinado.
\figura{0.4}{img/C.8-Button}{Widget: Button}{fig:NewFileCabbage}{}

Este es un ejemplo básico de \textbf{Button}:
\codigofuente{TeX}{Ejemplo de widget: Button}{codigo/ButtonEx}
Siendo sus funciones de funcionamiento análogo a lo visto previamente en el \textbf{CheckBox}.

\pagebreak
\subsection{Keyboard}
El widget \textbf{Keyboard} nos sirve para desplegar un teclado de formato occidental. Es realmente útil si queremos programar un instrumento sintetizador o si queremos tener una vía rápida de ejecutar samples de forma visual y práctica.
\figura{0.4}{img/C.9-Keyboard}{Widget: Keyboard}{fig:NewFileCabbage}{}

Este es un ejemplo básico de \textbf{Keyboard}:
\codigofuente{TeX}{Ejemplo de widget: Keyboard}{codigo/KeyboardEx}
Siendo \textbf{identchannel()} el identificador del widget respecto al código Csound de manera análoga al funcionamiento de \textbf{channel()}.

\subsection{Signal Display}
El widget \textbf{SignalDisplay} nos servirá, como su propio nombre indica, para mostrar señales de onda de manera visual.
\figura{0.4}{img/C.10-SignalDisplay}{Widget: Signal Display}{fig:NewFileCabbage}{}

Este es un ejemplo básico de \textbf{SignalDisplay}:
\codigofuente{TeX}{Ejemplo de widget: SignalDisplay}{codigo/SignalDisplayEx}
Siendo \textbf{displaytype()} el tipo de onda a mostrar de entre las siguientes opciones: 'spectrogram', 'spectroscope', 'waveform' y 'lissajous'. Y \textbf{signalvariable()} donde instroduciremos la variable con la onda que queremos que se muestre.

\section{Exportando nuestros instrumentos}\label{sec:ExportCabbage}

Una vez hemos terminado de programar el código de un instrumento, necesitamos alguna manera de hacer que ese código sea útil en el mundo real. Para ello Cabbage ofrece una serie de opciones de exportación del instrumento para que podamos usarlos donde queramos (ya sea sobre un software de terceros o de forma unitaria) y saquemos provecho de ellos.

Estas son las opciones de exportación de Cabbage:

\figura{0.6}{img/C.5-CabbageExport}{Las opciones de exportación del IDE}{fig:ExportCabbage}{}

En primer lugar averigüemos qué es un \textbf{VST}: 

Un plugin \textbf{VST} (\textbf{Virtual Studio Technology}) es una interfaz, en este caso digital, capaz de simular un instrumento o aportar un módulo entrada/salida para añadir efectos de sonido. Para usar un \textbf{VST} necesitamos un software compatible con este formato que sirva de base y desde el que ejecutemos nuestro \textbf{VST}. 
También es interesante destacar que \textbf{VST3} es el formato más moderno de \textbf{VST} con un código fuente más robusto y de fiar. El cual es también una de las opciones de exportación de Cabbage.

Será nuestro estándar de exportación, aunque Cabbage ofrece otras opciones como:

\begin{itemize}
 \item \textbf{VCV RackModule}: Siendo parecido al formato VST, es el tipo de módulo del que hablábamos en la sección de plantillas de creación de archivos en Cabbage. Es un formato de código abierto con una amplia comunidad y una documentación robusta en su web \url{https://vcvrack.com/Fundamental}
 \item \textbf{Standalone application}: Para generar un archivo ejecutable .exe que podremos usar en cualquier momento siempre y cuando tengamos Cabbage instalado en el equipo. Es interesante si por ejemplo codificamos un instrumento que pueda usarse como tal sin necesidad de otros softwares como un teclado digital.
 \item \textbf{FMOD Sound Plugin}: El formato usados por FMOD, \textbf{https://www.fmod.com/} y que está dedicado a la composición de sonido para juegos. Usando este formato, podremos generar plugins directamente compatibles con FMOD Studio.
\end{itemize}

En cualquier caso, Cabbage generará los archivos convenientes de exportación incluidos archivos .csd y .dll en el caso de sistemas windows. Será importante que mantengamos todos los archivos generados para un instrumento en el mismo directorio y así evitemos conflictos en la ejecución.



     % !TEX root = ../proyect.tex

\chapter{Fundamentos del Sonido}

\section{Introducción}\label{sec:intro}
Este capítulo tiene como función dar una breve introducción a la teoría física del sonido, en concreto a los conceptos fundamentalmente necesarios para entender los ejemplos expuestos en esta guía de Csound.
Se presenta por ello como capítulo anexado o capítulo extra de modo que sirva de referencia rápida en otras partes del documento y de manera que un lector con manejo en estos términos pueda saltar su contenido cómodamente.


\section{El Audio Digital}\label{sec:DigAud} 
Para definir el audio digital debemos empezar por saber qué es el sonido:\bigskip

\subsection{¿Qué es el sonido y cómo se transmite?}

\textbf{Sonido}: ``\textsl{Sensación producida en el órgano del oído por el movimiento vibratorio de los cuerpos, transmitido por un medio elástico, como el aire.}''\footnote{REAL ACADEMIA ESPAÑOLA: Diccionario de la lengua española, 23.ª ed., [versión 23.3 en línea]. <https://dle.rae.es> [05 de julio de 2020].}\bigskip

A ese movimiento vibratorio que se transmite y viaja por el medio podemos llamarlo ``Onda de Sonido''. Y la forma más simple de describir un movimiento vibratorio, es decir, la onda más simple de todas; es mediante la forma senoidal:

\figura{0.8}{img/S.1-senoidal}{Onda Senoidal}{fig:Senoidal}{}

Como sabemos, los medios transmisores están formador por moléculas que ocupan un determinado espacio. Podemos por lo tanto imaginar a una molécula que describe el movimiento vibratorio descrito anteriormente. Podríamos también dedir que cuando la molécula sobrepasa el punto inicial o punto 0 que definimos en la gráfica, la molécula está empujando al resto de moléculas que encuentra en su camino. De forma análoga, cuando la posición de la molécula tiene un valor menor al inicial decimos que la molécula está tirando del resto de moléculas de su entorno.

De esta manera se produce la transmisión del sonido.

\subsection{La onda de sonido y sus características}\label{sec:Ondas} 

Quedaba definida la onda de sonido en el apartado anterior. Si a continuación le añadimos la información de esa vibración de la que hablábamos es constante se producirá lo que llamamos ``Onda Periódica''.

Toda onda periódica posee 4 características:

\begin{itemize}
	\item \textbf{Periodo}: Es la cantidad de tiempo que tarda la forma de la onda en repetirse, lo llamaremos T y lo expresaremos en segundos.
	 \item \textbf{Amplitud}: Distancia máxima de los puntos de la onda respecto a la posición de eje Y (eje ``Tiempo'' en la figura). Podemos definirla también como la fuerza con la que las moléculas del medio consiguen empujar o tirar del resto de moléculas de su entorno.
\figura{0.6}{img/S.2-Periodo}{Periodo y Amplitud}{fig:Periodo}{}
    \item \textbf{Frecuencia}: La frecuencia de una onda expresa la cantidad numérica de veces que repite su movimiento durante un tiempo determinado. Si la definimos respecto al periodo decimos que es la cantidad numérica de periodos (o repeticiones de la forma de la onda) que ocurren durante un segundo. Se mide en Hercios o Hz, la representaremos con f y podremos calcularla fácilmento puesto que es la inversa del periodo:
    \begin{itemize}
    \item Frecuencia = 1/Periodo
    \item Periodo = 1/Frecuencia   
    \end{itemize}
    \item \textbf{Longitud de Onda}: Se trata de la distancia que va del punto inicial al punto final del recorrido de la onda marcada por un periodo. Se mide en metros.
    \item \textbf{Fase}: Punto de partida de la onda. Podemos observar al representar la onda en un gráfico y fijarnos en que el valor inicial del eje Y que no tiene que ser nesesariamente 0.
\end{itemize} 

\subsection{El Sampleo y Sample Rate}\label{sec:sr} 

Para representar de manera digital una onda de sonido acústica necesitamos convertir sus valores analógicos a digitales. Esta onda tendrá un valor distinto por cada instante de tiempo y para conseguir recoger de alguna manera estos datos en un computador necesitaremos el concepto de sampleo.

El sampleo consiste en recoger un número determinado de valores en formato digital de una onda sonora por cada segundo de duración.

\figura{0.8}{img/S.3-Sampleo}{Ejemplo de sampleo de una onda}{fig:Sampleo}{}

Aquí entra el concepto de ``sample rate'', que determina la frecuencia de muestreo en la recogida de datos de la onda. En la figura anterior podemos observaruna misma onda de sonido a la que se le realiza un sampleo primero con sample rate menor (izquierda), y con un sample rate mayor (derecha).

Por último debe destacarse que un sample rate mayor no implica necesariamente una coversión digital más feaciente para nosotros a un nivel pragmático puesto que el nivel máximo de frecuencia de datos sonoros que puede captar en oído humano ronda los 20Khz. 

Contamos además con que debe respetarse el \textbf{Teorema de muestreo de Nyquist-Shannon}:\bigskip

\textsl{``Para representar una onda de manera digital que contenga frecuencias de hasta X Hz, es necesario usar un sample rate de al menos 2X muestras por segundo.''}\bigskip

De otra manera, los valores digitales no representarían la onda de manera correcta dando lugar al aliasing y al muestreo incorrecto.

Se muestra a continuación una figura de ejemplo de sampleo en la que se toma un sample rate de 40000Hz. De la primera onda, que es de 10KHz observamos que recogemos los datos suficiente cada segundo como para captar toda la información contenida en ella. De hecho podemos observar también que con un sample rate de 20KHz en lugar de 40KHz también captaríamos toda la información necesaria como para recoger la onda al completo en su formato digital cumpliéndose así el Teorema de muestreo de Nyquist-Shannon.

Sin embargo la onda del segundo gráfico tiene una frecuencia de 30KHz y esto implica que con un sample rate de 40KHz como el mostrado, daría resultado una conversión de onda errónea. Dando incluso para este ejemplo una muestra de onda digital idéntica a la del primer gráfico. Sería necesario un sample rate de al menos 60KHz para recoger feacientemente los datos y conseguir una conversión digital satisfactoria de la onda.\pagebreak

\figura{0.8}{img/S.4-Aliasing}{Un mismo sample rate para ondas distintas}{fig:difSamples}{}

\section{Conceptos interesantes}\label{sec:Conceptos} 

Hablemos a continuación de algunos conceptos no necesariamente intrínsecos al sonido digital pero interesantes para entender mejor los conceptos necesarios para el uso adecuado de Csound.

\subsection{El decibelio}\label{sec:db} 

El decibelio o `db'\ es una unidad de medida que representa la intensidad de un sonido. Es la décima parte de un belio y siempre que hablamos de decibelio lo hacemos respecto a un valor de referencia preestablecido de intensidad, normalmente el marcado por el límite por debajo del oído humano en su capacidad para oír: \( I_{0} = 10^{-12}W/m^2 \) que se da en los 1000Hz.

La fórmula para calcular los decibelios es: \( 10*log_{10}* \frac{I}{I_{0}} \), es por tanto una fórmula logarítmica que depende de su valor de referencia 
\( I_{0} \). Para una relación \( \frac{I}{I_{0}} \) de 10 tenemos 10db, para una relación de 100 tendremos 20db, para 1000 30db, etc...

Un dato útil a tener en cuenta es que al doblar la amplitud de una onda de sonido obtenemos una diferencia de +6db. De manera análoga si partimos por la mitad el valor de la amplitud de la onda obtenemos un cambio de -6db.

\subsection{El ADSR}\label{sec:ADSR} 

\textbf{ADSR} son las siglas de \textbf{A}ttack, \textbf{D}ecay, \textbf{S}ustain y \textbf{R}elease en una onda y conforman la opción más común de envelope o envolvente sonoro, es decir, proporcionan parámetros para poder controlar una onda de sonido.

Veamos cada uno de estos parámetros para entenderlos en conjunto:

\begin{itemize}
    \item \textbf{Attack}: El Attack o Ataque sería lo ocurrido antes de que la onda decaiga y se estabilice. Por ejemplo, un golpe de platillo produce un sonido con mucho ataque, el sonido de una nota tocada en una flauta dulce tendría normalmente poco ataque.
    \item \textbf{Decay}: El decay o decaimiento es lo sucedido entre el ataque ensu máximo punto y la fase estable de la onda. Al rasgar las cuerdas de una guitarra con una púa se produce un sonido con bastante decay.
    \item \textbf{Sustain}: El sustain o sostenibilidad es lo referido a la parte estable de la onda, su duración e intensidad máxima. Un golpe de caja tiene poco sustain, una nota tocada al aire en la cuerda de un bajo eléctrico tiene mucho sustain.
    \item \textbf{Release}: El release es la parte de la onda comprendida entre la fase estable o de sustain y la llegada al valor 0 de intensidad. Los instrumentos de cuerda tienen por lo gerenar una fase de release notable.
\end{itemize}

\figura{0.8}{img/S.5-ADSR}{El envolvente ADSR de una onda}{fig:difSamples}{}

Es común en el sonido digital poder modificar estos parámetros a placer en instrumentos como los sintetizadores o en prácticamente cualquier instrumento con componentes eléctricos.

\subsection{El Cutoff y la resonancia}\label{sec:Cutoff} 

Veamos este par de conceptos útiles para completar y mejorar nuestro sonido:

\begin{itemize}
    \item \textbf{Cuttoff}: Se trata de un filtro de frecuencias. Normalmente se usa para bloquear determinados rangos de frecuencias altas, siendo esto de tipo LP (LOW PASS) aunque en instrumentos modernos pueden encontrar Cutoffs de tipo HP (HIGH PASS) que bloquearían en este caso frecuencias más bajas de lo deseado.
    \item \textbf{Resonancia}: Es el efecto que ocurre por ejemplo cuando la mesa que sujeta unos altavoces en marcha empieza a vibrar mientras algunos sonidos son producidos, querrá decir que el sonido que sale de esos altavoces producen una frecuencia coincidente con la frecuencia de resonancia de la tabla de la mesa y el sonido por tanto se ve amplificado. En el audio digital puede encontrarse la resonancia como un efecto más a añadir a nuestro arsenal de producción musical. 
\end{itemize}





     % !TEX root = ../proyect.tex

\chapter{Bibliografía Comentada}

Se presenta la bibliografía comentada de cada referencia usada en el presente documento.

\section{Csound FLOSS Manual}

 \begin{itemize}
 \item \textbf{Tipo de fuente}: Libro online
 
 \item \textbf{Última actualización}: Marzo de 2015
 
 \item \textbf{Dificultad}: Recomendado para principiantes.
 
 \item \textbf{Autor/es}: Joachim Heintz, Iain McCurdy, Andres Cabrera, Alex Hofmann, Alexandre Abrioux, Rory Walsh, Martin Neukom, Jim Aikin, Jacques Laplat, Menno Knevel, Bjorn Houdorf, Christopher Saunders, Oeyvind Brandtsegg, Oscar Pablo di Liscia, Peiman Khosravi, Steven Yi, Stefano Bonetti, Victor Lazzarini, Ed Costello, François Pinot, Tarmo Johannes, Nicholas Arner, Nikhil Singh, Richard Boulanger, Michael Gogins, Anton Kholomiov.
 \end{itemize}

\subsection{Referencia}

Comunidad de desarrolladores del lenguaje Csound (marzo de 2015). Csound FLOSS Manual. FLOSS Manuals. \url{http://write.flossmanuals.net/csound/preface/}

\subsection{Comentario}

El \textbf{Csound FLOSS Manual} es la principal fuente bibliográfica de este documento\footnote{Se ha sido especialmente extenso en el comentario de esta referencia por ser la principal fuente bibliográfica.}. Ha sido escrito por el núcleo de la comunidad de desarrolladores del código abierto de Csound y podría decirse sin miedo a equivocarse que es la fuente más completa y accesible para aprender sobre el lenguaje en la red.

Su estructura es la de un libro electrónico y la gran mayoría de sus capítulos cuenta con ejemplos de código completos e ilustrativos. Tiene además un capítulo dedicado por completo a explicar conceptos básicos sobre el sonido y el mundo de la edición sonora, aportando una buena base sólida del conocimiento previo que se recomendaría tener antes de probar un lenguaje dedicado al sonido.

Con todo y por estar realizado de mano directa por algunos de los principales autores del código fuente de Csound, \textbf{Csound FLOSS Manual} resulta ser un compendio actual de todos los conocimientos del lenguaje y es por ello que se recomienda encarecidamente su estudio y comprensión si se quiere aprender realmente los fundamentos de Csound.

\textbf{Csound FLOSS Manual} ha servido como base al presente documento tanto en su estructura como en su metodología para presentar los conocimiento.

En ocasiones \textbf{Csound FLOSS Manual} puede resultar algo tedioso precisamente por la completitud de sus ejemplos y explicaciones, cosa que se ha tratado de solventar en el presente documento al simplificar alguna explicaciones, pero es precisamente por ello que lo debido es recurrir al \textbf{Csound FLOSS Manual} para empezar a profundizar realmente en lo aprendido tras revisar los conocimientos aquí mostrados.

\subsection{Estructura}

\textbf{Csound FLOSS Manual} tiene 13 capítulos principales más 2 capítulos extra que resumiremos a continuación para tener una referencia útil acerca de dónde buscar para profundizar en cada concepto:

\begin{itemize}
 \item \textbf{01 BASICS}: Conceptos fundamentales sobre el sonido y su procesamiento. Muy útil incluso para el que sólo esté interesado en el mundo del sonido y no necesariamente en Csound.
 \item \textbf{02 QUICK START}: Información más básica sobre el lenguaje y sus IDEs. Cómo ejecutar programas, cómo exportarlos,  etc...
 \item \textbf{03 CSOUND LANGUAGE}: Fundamentos del lenguaje. Su sintaxis y las diferentes propiedades básicas como tipos de variables o funciones.
 \item \textbf{04 SOUND SYNTHESIS}: Conceptos físicos aplicados a la síntesis del sonido. Posee ejemplos más complejos para complementar las explicaciones algo más academicas.
 \item \textbf{05 SOUND MODIFICATION}: Capítulo dedicado principalmente a las capas de envoltura y efectos de sonido mediante filtros.
 \item \textbf{06 SAMPLES}: Capítulo dedicado a la lectura y escritura de archivos y a su consecuente aplicación en lo referente a los datos del sonido.
 \item \textbf{07 MIDI}: Dedicado a lo referente al MIDI (Musical Instrument Digital Interface) dando una extensa explicación acerca de cómo vincular nuestros instrumentos físicos o virtuales a nuestro código.
 \item \textbf{08 OTHER COMMUNICATION}: Capítulo corto pero interesante acerca de cómo combinar Csound con OSC y projectos con Arduino.
 \item \textbf{09 CSOUND IN OTHER APPLICATIONS}: Capítulo dedicado a explicar cómo combinar Csound con otros lenguajes y tecnologías dedicadas como PureData o Ableton Live.
 \item \textbf{10 CSOUND FRONTENDS}: Nos da una revisión media sobre los principales entornos de programación entre los que podemos elegir para usar Csound.
 \item \textbf{12 CSOUND AND OTHER PROGRAMMING LANGUAGES}: Como el propio título indica, se nos explica cómo y con qué sintaxis podemos compilar código Csound en diferentes lenguajes como Python  o Haskell.
 \item \textbf{13 EXTENDING CSOUND}: Capítulo corto que nos da una pequeña introducción acerca de cómo aportar al código abierto del lenguaje mediante, por ejemplo, la creación de nuevos opcodes.
 \item \textbf{OPCODE GUIDE}: Capítulo extra que aporta información más extensiva acerca del uso y funcionamiento de los opcodes.
 \item \textbf{APPENDIX}: Por último el apéndice, que aporta recomendaciones de nomenclatura, un glosario corto y una librería de enlaces con webs de información intereseante sobre Csound y el mundo del sonido.
\end{itemize}

\section{The Canonical Csound Reference Manual}

 \begin{itemize}
 \item \textbf{Tipo de fuente}: Manual online
 
 \item \textbf{Última actualización}: Enero de 2020
 
 \item \textbf{Dificultad}: Necesario conocimiento previo.
 
 \item \textbf{Autor/es}: Barry Vercoe, Comunidad de Csound.
 \end{itemize}

\subsection{Referencia}

Comunidad de Csound (Enero de 2020). The Canonical Csound Reference Manual. Csound. \url{https://csound.com/docs/manual/index.html}

\subsection{Comentario}

\textbf{The Canonical Csound Reference Manual} es el manual más extenso de información sobre Csound en la red. Es por ello algo menos accesible para principiantes en el lenguaje pero el mejor compendio de referencias si necesitamos información acerca de, por ejemplo, un opcode concreto. Este manual ha sido creado por los desarrolladores del lenguajes Csound y se actualiza conjuntamente con el propio lenguaje para reflejar las nuevas características del código fuente.

En referencia a este documento, ha servido para completar y complementar conocimientos en el uso y sintaxis concretos de algunos métodos y opcodes, y descripción de algunos conceptos del lenguaje que sólo pueden encontrarse en el propio manual.
Se recomienda su uso como guía de referencia siempre que se sepa previamente qué se está buscando.

Su contenido se divide en tres partes principales, una primera describiendo conceptos de Csound, una segunda extendiendo este conocimiento e introduciendo conceptos como la generación y edición de señales, y una tercera a modo de biblioteca de referencia de todos los opcodes (incluyendo los obsoletos) disponibles en Csound. Posee además una biblioteca de archivos descargables con cientos de ejemplos de programación y apartados con datos útiles como una tabla de conversión de valores de onda a nota musical.

\subsection{Estructura}

 \begin{itemize}
 \item \textbf{I. Overview}: Da una base fundamental del uso del lenguaje, como punto a favor posee un enlace a una página de la misma guía cada vez que se menciona algún opcode o palabra reservada por lo queda bien estructurado.
 
 \item \textbf{II. Opcodes Overview}: Ofrece información extensa acerca del uso de los principales opcodes de Csound dividiendo el capítulo según sus tipos (De control de instrucciones, de edición de señales de audio, relacionadas con el MIDI, etc...)
 
 \item \textbf{III. Reference}: Biblioteca de referencia de los diferentes opcodes y operadores del lenguaje que se recomienda tener siempre a mano. El resto de manual hace constante referencia a esta sección.
 
 \item \textbf{Apéndices}: Ofrece información útil sobre diversos temas relacionados al sonido o a la sintaxis concreta de Csound.
 \end{itemize}

\section{Cabbage Docs}

 \begin{itemize}
 \item \textbf{Tipo de fuente}: Documentación online
 
 \item \textbf{Última actualización}: Febrero de 2020
 
 \item \textbf{Dificultad}: Recomendado para principiantes.
 
 \item \textbf{Autor/es}: Rory Walsh, Iain McCurdy, Gordon Boyle.
 \end{itemize}

\subsection{Referencia}

Walsh R.(febrero de 2020). Cabbage Docs. Cabbage. \url{https://cabbageaudio.com/docs/introduction/}

\subsection{Comentario}

\textbf{Cabbage Docs} es la fuente bibliográfica de todo conociemiento referente al uso del IDE Cabbage y a sus Widgets. Ha sido escrita por Rory Walsh, principal desarrollador y autor de Cabbage, por lo que puede considerarse una fuente fiable de conocimiento.
Entre sus secciones ofrece una corta introducción a Csound que se recomienda usar como repaso al lenguaje, una sección dedicada a explicar el uso de Cabbage como IDE que cuenta con buenos ejemplos prácticos, y una sección a modo de biblioteca de referencia de uso de los diferentes Widgets existentes la cual se recomienda usar como API de ejemplos de uso específico del IDE.

\subsection{Estructura}

\textbf{Cabbage Docs} tiene cuatro secciones:
 \begin{itemize}
 \item \textbf{Beginners(Csound)}: Da una introducción al lenguaje Csound.
 
 \item \textbf{Using Cabbage}: Da una introducción a cómo sacar provecho del IDE Cabbage al usar Csound
 
 \item \textbf{Advanced Features}: Extiende algo más sobre el uso de Csound y sus conceptos de uso avanzado.
 
 \item \textbf{Cabbage Widgets}: Sirve de biblioteca de referencia de los diferentes Widgets aportados por Cabbage.
 \end{itemize} 
 
\section{Página oficial de Csound}
 
\section{Canal youtube Steven Yi y Github de csound-live-code}
%     \chapter{Definici\'on de objetivos}\label{defobjetivos}
Estos son los objetivos generales del presente TFG. Se agrega a cada uno de ellos una serie de objetivos específicos:

\begin{itemize}
\item Dar a conocer en mayor medida el lenguaje de programación Csound.
	\begin{itemize}
	\item Expresar el contenido de los capítulos en formato divulgativo.
	\item Transmitir de manera feaciente las ventajas y desventajas de Csound en un capítulo introductorio.
	\end{itemize}
\item Proporcionar una guía de aprendizaje introductorio al lenguaje Csound.
	\begin{itemize}
	\item Mostrar los contenidos en orden por dificultad, para que se pueda seguir el documento linealmente.
	\item Mostrar, al menos superficialmente, varios ámbitos del lenguaje como el ``Live Coding'', la sintaxis base del lenguaje, y otras posibilidades. 
	\item Dar una introducción práctica al IDE Cabbage.
	\end{itemize}
\item Proporcionar un documento de consulta rápida de conceptos del lenguaje Csound.
	\begin{itemize}
	\item Mostrar los contenidos de forma unitaria, proporcionando además un índice ordenado y pragmático.
	\item Proporcionar anexos con contenido relacionado como los fundamentos del sonido y otros conceptos útiles.
	\end{itemize}
\item Proporcionar ejemplos prácticos de programación usando el lenguaje Csound a modo de demostración capacitiva del lenguaje.
	\begin{itemize}
	\item Acompañar la mayoría de lecciones con al menos un ejemplo de código real.
	\item Dar referencias a otros portales de contenido de Csound que cuenten con una librería de ejemplos prácticos.
	\end{itemize}
\item Compilar una lista de referencias a portales de contenido de Csound de manera ordenada y comentada.
	\begin{itemize}
	\item Realizar una bibliografía comentada, centrada especialmente en las principales fuentes del documento sirviendo así de guía para saber por dónde seguir para extender conocimientos.
	\end{itemize}
\end{itemize}
%     \chapter{An\'alisis de requisitos, dise\~no e implementaci\'on}\label{requisitos}




\section{Dise\~no e implementaci\'on}

%     \chapter{An\'alisis de antecedentes y aportaci\'on realizada}\label{analanteced}
 
%     \chapter{An\'alisis temporal y costes de desarrollo}\label{anatemporal}

\section{An\'alisis temporal}
	
\section{Costes de desarrollo}		
	

	
%     \chapter{Comparación con otras alternativas}\label{alternativas}













	








	
	 





%     \chapter{Pruebas}\label{pruebas}



	



%     \chapter{Manual}\label{manual}



\backmatter

% \chapter{Apéndices}\label{apendices}
%\bibliographystyle{sousa5}

%\bibliographystyle{apacite}
%\bibliography{pfcbib}

\end{document}