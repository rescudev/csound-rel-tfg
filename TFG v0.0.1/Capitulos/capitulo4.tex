% !TEX root = ../proyect.tex

\chapter{Conceptos Avanzados}

\section{El opcode ftgen}

el opcode \textbf{ftgen} es muy recurrente. Nos sirve para generar una tabla de scores de entre nuestros instrumentos.

Su sintaxis base es:

\codigofuente{TeX}{Sintaxis base de ftgen}{codigo/ftgenSyntax}

Siendo \textbf{gir} una tabla de al menos 100 posiciones, \textbf{ifn} el número de la tabla. El resto de valores se corresponden con los argumentos \textbf{p2}, \textbf{p3}, \textbf{p4} y \textbf{p5} del \textbf{f Statement} que explicaremos a continuación.

\subsection{El f statement}

Coloca valores en una tabla de scores para usar nuestros instrumentos. Está implícita en \textbf{ftgen}.

Esta es la sintaxis base:

\codigofuente{TeX}{Sintaxis base del f statement}{codigo/fStateSyntax}

Siendo \textbf{p1} el número de tabla, \textbf{p2} tiempo de activación para generar datos, \textbf{p3} tamaño de la tabla, \textbf{p4} nombre de la rutina \textbf{GEN} de generación de datos y \textbf{p5 ... PMAX} que dependerán del \textbf{GEN} que estemos usando.

Este es un ejemplo en el que rellenamos una tabla (f1) de tamaño 5 con ceros: \textbf{( f1 0 5 2 0\ )}

\section{Delay y Feedback}

Empecemos por explicar en qué consisten estos efectos. 

\subsection{El Delay}
Podemos definir el \textbf{Delay} como un buffer de repetición de nuestra onda que va poniendo en cola los sonidos que van entrando y según las características de nuestro \textbf{Delay} los saca de una manera determinada, conformando así algunos de los efectos más conocidos en la producción musical. 

El uso más básico del \textbf{Delay} es el clásico efecto ``eco'' pero también puede servirnos para implementar el Chorus, Flanging, Cambio tonal y otros filtros de onda.

Respecto a Csound existen varios opcodes dedicados al \textbf{Delay}, como \textbf{delayr} y \textbf{delayw}.

Empecemos por la diferencia en los nombres de estos opcodes que se usan en conjunto puesto que \textbf{delayr} es de lectura (read) y \textbf{delayw} es de escritura (write) del buffer de delay. 

Veamos un ejemplo básico de uso:

\codigofuente{TeX}{Sintaxis delayr y delayw}{codigo/delaySyntax}

Donde \textbf{aSignIn} es la señal de entrada a la que queremos aplicar el delay y \textbf{aSignOut} es la señal de salida a la que ya se ha aplicado el efecto.
El '1' que aparece como argumento es la cantidad de segundos que mide el buffer, en este caso un segundo y el hecho de definir aquí la longitud del buffer implica que tengamos que definir primero la variable de salida antes y después la variable de entrada, lo cual es en principio un poco confuso.

\subsection{El Feedback}

El delay tal y como lo hemos definido anteriormente acaba pareciendo una simple única repetición atenuada del sonido de entrada. Para encontrar el clásico sonido al que en el mundo de la edición musical nos referimos como delay debemos conocer el concepto de \textbf{feedback}.

El \textbf{feedback} será un valor residual de la señal del buffer que usaremos como parámetro de entrada del delay. Lo normal es que este valor decrezca con cada paso por el buffer de manera que el sonido acabe pareciendo cada vez más distante hasta apagarse por completo y no se produzca un bucle infinito.

Respecto a Csound podremos implementar este concepto de manera sencilla con una variable.

\subsection{Ejemplo del efecto Delay}

Veamos a continuación el código de un efecto delay con feedback y analicémoslo con el objetivo de comprenderlo mejor:

En primer lugar podemos observar tres opcodes que no habíamos visto hasta el momento. \textbf{loopseg} en la línea 16, \textbf{randomh} en la línea 17 y \textbf{interp} en la línea 18. 

Expliquemos brevemente sus funcionamientos:

\begin{itemize}
 \item \textbf{loopseg}: Genera una señal de control que puede usarse como envelope o envoltura. Tantovariable de salida como variables de entradas serán de tipo k-rate.
 \item \textbf{randomh}: Genera valores aleatorios al igual que el opcode \textbf{random} pero mantiene estos valores guardados durante un periodo determinado de tiempo.
 \item \textbf{interp}: Sirve para convertir una variable de tipo k-rate a una variable de tipo a-rate.
\end{itemize}

Una vez sabido el funcionamiento básico de cada opcode presente veamos paso por paso lo que  se hace en el ejemplo:

\begin{itemize}
 \item \textbf{Línea 13}: Hacemos uso de \textbf{ftgen} para tener preparada nuestra onda senuidad.
 \item \textbf{Líneas 15 a 19}: Definimos las primeras líneas de nuestro instrumento. El uso de estas líneas es el de acabar generando \textbf{aSig}, es decir, nuestra onda de entrada a la que aplicaremos el delay.
 \item \textbf{Línea 21}: Definimos \textbf{iFback} que tal y como explicamos en secciones anteriores hará las veces de valor de feedback, en este caso 0.7 (de ser un valor menor nuestro delay se apagaría antes).
 \item \textbf{Líneas 22 a 26}: Hacemos uso de \textbf{delayr} y \textbf{delayw}. Si prestamos especial atención a la línea 23 veremos que nuestro valor de entrada, \textbf{(aSig(aBufOut*iFdback))}, es efectivamente una operación en la que participan nuestra señal, su propio valor de salida (siguiente en el buffer) y el valor de feedback que acabará determinando el tiempo restante del delay.
 
 Como vemos también en la línea 25, el sonido que nuestro instrumento producirá será una mezcla de la onda de salida con el valor de buffer.
\end{itemize}

\codigofuente{TeX}{Ejemplo completo del efecto Delay}{codigo/delayCompleto}

Se invita al lector a que experimente con diversos valores en las variables mencionadas para comprender en mayor medida el concepto de delay y que posteriormente visite las referencias del documento en caso querer profundizar.
\pagebreak

\section{FM: La modulación de frecuencia}

La modulación de frecuencia, tambien conocida como \textbf{FM (Frequency Modulation)} es la categoría de los efectos de sonido basados en la modificación de la frecuencia de la onda de sonido original. Un claro ejemplo de ello es el efecto de vibrato.

\subsection{El vibrato}

El efecto \textbf{Vibrato} en el sonido consiste en modificar la frecuencia (por tanto el tono) de la onda sonora de forma periódica y constante  con valores cercanos al original. Se crea así una sensación de ``vibración''\ del sonido. 

Veamos un ejemplo de \textbf{vibrato} implementado en Csound.
\codigofuente{TeX}{Ejemplo completo del efecto Vibrato}{codigo/vibratoCompleto}

Prestemos especial atención a las líneas 12 y 13 de la figura anterior. En primer lugar la línea \textbf{(aMod poscil 10, 5, 1)} produce una onda de 10Hz de amplitud y 5Hz de frecuencia. Esta será nuestra onda moduladora, la que aplicaremos sobre el sonido al que queremos añadir el efecto vibrato.

En segundo lugar tenemos la línea \textbf{(aCar poscil 0.3, 440+aMod, 1)} que produce una onda de 440Hz de frecuencia y a la que sumamos el valor de \textbf{aMod}, nuestro modulador, que afecta al valor de la frecuencia de \textbf{aCar} haciéndolo oscilar periódicamente entre 450Hz y 430Hz.
Si sustituimos la línea anterior por: \textbf{(aCar poscil 0.3, 440, 1)} comprobaremos que deja de producirse el efecto \textbf{vibrato}.

Como podemos observar el efecto vibrato simple puede lograrse fácilmente en Csound, pero se invita a  experimentar con los valores de \textbf{aMod} para comprobar cómo afecta una onda moduladora con distintos valores de amplitud y frecuencia.
\pagebreak

\section{AM: La modulación de amplitud}

La modulación de la amplitud de onda o \textbf{AM (Amplitude Modulation)} consiste en hacer modificaciones en la amplitud de una onda, esto provoca lógicamente un aumento o disminución del volumen del sonido producido. Uno de los ejemplos más claros de efecto basado en el \textbf{AM} es el efecto trémolo.

\subsection{El trémolo}

Es fácil confundir el efecto trémolo con el vibrato pero como acabamos de describir, el efecto trémolo se basa en modificaciones de amplitud y no de frecuencia de la onda. No obstante, para implementar este efecto en Csound procederemos de manera similar haciendo uso de una variable moduladora(una segunda onda) que aplicaremos sobre sobre el sonido al que queremos aplicar el trémolo. 

Veamos un ejemplo:

\codigofuente{TeX}{Ejemplo completo del efecto Trémolo}{codigo/tremoloCompleto}

Empecemos por explicar el opcode \textbf{expseg} que vemos por primera vez. Este opcode con sintaxis \textbf{(ares expseg ia, idur1, ib)} genera un valor que cambia exponecialmente a lo largo del tiempo, es decir, nos da un valor inicial \textbf{ia} que aumenta su valor exponencialmente a los largo de \textbf{idur1} segundos hasta llegar al valor \textbf{ib}. Aunque este valor sobrepase a \textbf{ib}, seguirá aumentando en el mismo ratio de exponecialidad hasta que paremos su ejecución. Es debido a este opcode y su valor \textbf{aRaise} en el ejemplo que podemos oír un aumento de la ``rapidez'' del trémolo a medida que pasa el tiempo.

Observemos detenidamente las líneas 14, 16 y 17 de la figura. Tenemos por una parte la onda \textbf{aModSine} que será nuestro modulador de amplitud y por otro nuestra onda original y simple de 440Hz \textbf{aCarSine}. Como vemos en la línea 17 aplicaremos un cálculo simple de multiplicación para que las amplitudes, \textbf{0.5} de \textbf{aModSine} y \textbf{0.3} de \textbf{aCarSine} se sumen y oigamos el efecto trémolo a una frecuencia de \textbf{aRaise}. Se invita al lector a ejecutar este mismo ejemplo sustituyendo la variable \textbf{aRaise} en la línea 14 por el valor fijo 1 y se podrá observar cómo el efecto trémolo tendrá una duración constante de exactamente un segundo.

Por último en la línea 15 tenemos la variable \textbf{aDCOffset} que determina el DC Offset de la onda. Veamos con un poco más de profundidad este concepto.

\subsection{DC Offset}
El \textbf{DC Offset} es un valor añadido a la onda que en términos gráficos podemos decir que la ``desplaza'' respecto del valor 0. 
 \figura{0.6}{img/4.1-Offset}{El DC Offset}{fig:Offset}{}
 
El DC Offset tiene muchos usos cuando es intencionado como el de prevenir el clipping, pero en algunos casos se produce de forma indeseada por causas ajenas.
 
\section{RM: La modulación en anillo}

La modulación en anillo o \textbf{RM (Ring-Modulation)} es un caso especial de modulación de amplitud en el que no hacemos uso del \textbf{DC Offset}. De esta manera diremos que la amplitud variará entre +1 y -1 (respecto a 0dbfs).

Para tener un ejemplo básico de esto basta con que volvamos a la figura de código 4.6 y demos a \textbf{aDCOffset} un valor de 0.

\section{WAVESHAPING}
http://write.flossmanuals.net/csound/e-waveshaping/

\section{Filtros de onda}
Los \textbf{filtros de onda} son uno de los recursos más usados en la edición musical y sonora. Consisten principalmente en filtrar o cortar determinadas frecuencias en un sonido dejando pasar al resto. Hablemos de las tres principales categorías de \textbf{filtros de onda}.
Con el concepto ``frecuencia cutoff'' nos referiremos a la frontera entre las frecuencias que pasan y las que corta el filtro.

\subsection{Filtros Lowpass}

Los filtros \textbf{Lowpass} dejan pasar las frecuencias de onda más bajas y filtran las más altas. Para implementar este tipo de filtros en Csound existen diversos opcodes, veamos un ejemplo con tres de ellos: \textbf{tone}, \textbf{butlp} y \textbf{moogladder}.

\codigofuente{TeX}{Opcodes de filtro Lowpass}{codigo/lowpassCompleto}

En la figura anterior se definen tres instrumentos que empiezan generando la misma onda pero a la que se acaba aplicando un opcode de filtro \textbf{lowpass} distinto.

\begin{itemize}
 \item \textbf{tone}: Este opcode se caracteriza por usar una frecuencia de cutoff no demasiado estricta, es decir, que actúa como atenuador de las frecuencias más adyacente al cutoff en lugar de cortarlas abruptamente.
 \item \textbf{butlp}: Usa en cambio un cutoff más agresivo que \textbf{tone} por lo que quedarán menos residuos de la frecuencias más altas en nuestro sonido.
 \item \textbf{moogladder}: Este opcode se basa en filtros analógicos usados en sistetizadores de la marca ``Moog''. La mejor forma de describirlo sería por tanto en referencia a su contraparte original.
\end{itemize}

La mejor forma de comprender las diferencias entre estos opcodes es ejecutando el ejemplo anterior y tratando de analizar la diferencia sonora entre instrumentos.

\subsection{Filtros Highpass}

Los filtros \textbf{Highpass} dejan pasar las frecuencias de onda más altas y filtran las más bajas. Para implementar este tipo de filtros en Csound existen diversos opcodes, veamos un ejemplo con tres de ellos: \textbf{atone}, \textbf{buthp} y \textbf{bqrez}.

\codigofuente{TeX}{Opcodes de filtro Highpass}{codigo/highpassCompleto}

De forma análoga a la saección anterior, se definen tres instrumentos. Veamos sus opcodes de filtro \textbf{Highpass}.
\pagebreak
\begin{itemize}
 \item \textbf{atone}: De manera análoga a \textbf{tone}, este opcode se caracteriza por usar una frecuencia de cutoff no demasiado estricta, es decir, que actúa como atenuador de las frecuencias más adyacente al cutoff en lugar de cortarlas abruptamente.
 \item \textbf{buthp}: De manera análoga a \textbf{butlp}, \textbf{buthp} usa un cutoff más agresivo que \textbf{atone} por lo que quedarán menos residuos de la frecuencias más bajas en nuestro sonido.
 \item \textbf{bqrez}: Este es un opcode con múltiples modos para seleccionar, en nuestro ejemplo usamos ``mode:1'' pero se invita a experimentar con otros modos para comprender mejor este opcode.
\end{itemize}

\subsection{Filtros Bandpass}
Este tipo de filtros dejan pasar frecuencias comprendidas en una determinada ``banda'', es decir, será necesario determinar cómo de ancha será esa banda conociendo sus límites superior e inferior. Veamos la implementación de este filtro en Csound con dos opcodes: \textbf{reson} y \textbf{butbp}.
\codigofuente{TeX}{Opcodes de filtro Bandpass}{codigo/bandpassCompleto}
\begin{itemize}
 \item \textbf{reson}: Filtro de banda basado en la resonancia.
 \item \textbf{butbp}: La versión de \textbf{bandpass} de los opcode \textbf{but--}. Con un cutoff relativamente agresivo. 
\end{itemize}

\section{Reverberación}
La \textbf{reverberación} o \textbf{reverb} es el conocido efecto sonoro que se produce cuando hablamos en una habitación vacía o se produce un ruido en espacios cerrados muy amplios como iglesias o catedrales. 

El efecto físico asociado consiste en una mezcla entre los ecos del propio sonido que se sobreponen unos sobre otros por rebotar continuamente en las paredes del habitáculo.

El opcode más común para conseguir el efecto \textbf{reverb} en Csound es \textbf{freeverb}, veamos un ejemplo de código:

\codigofuente{TeX}{Opcodes de filtro Bandpass}{codigo/revCompleto}

Vemos por primera vez el opcode \textbf{pinkish}, que genera una onda de ruido rosa o \textbf{pink noise}.
 
En la figura de código anterior se definen dos instrumentos. El primero con el que generamos los golpes sonoros y el segundo con el generamos el efecto \textbf{reverb} del sonido anterior gracias al opcode \textbf{freeverb}. Si observamos mejor el instrumento 1 vemos que definimos una señal de audio sencilla \textbf{aSig} y definimos más tarde las características del \textbf{reverb} con la variable global \textbf{gaRvSend}. En el segundo instrumento es donde hacemos uso de \textbf{freeverb}, veamos su sintaxis en profundidad: 

\textbf{(aoutL, aoutR freeverb ainL, ainR, kRoomSize, kHFDamp)}

Siendo \textbf{ainL} y \textbf{ainR} las señales de entrada que, como en nuestro ejemplo, suelen ser la misma. El argumento \textbf{kRoomSize} representa el tamaño de la sala en la que se produce el reverb, es un valor de 0 a 1 y determina directamente la duración del reverb. Por último \textbf{kHFDamp} es la variable de control de altas frecuancias, a más se acerque a 1 este valor más se atenuarán las frecuencia mayores y con un valor de 0 la atenuación durante el efecto reverb será el mismo para todas las frecuencias.

Se invita a experimentar con los valores de \textbf{iRvbSendAmt}, \textbf{kroomsize} y \textbf{kHFDamp} del ejemplo para comprender mejor el efecto reverb y el opcode freeverb.