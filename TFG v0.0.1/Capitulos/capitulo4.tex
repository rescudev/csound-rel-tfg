% !TEX root = ../proyect.tex

\chapter{Conceptos Avanzados}

\section{El opcode ftgen}

el opcode \textbf{ftgen} es muy recurrente. Nos sirve para generar una tabla de scores de entre nuestros instrumentos.

Su sintaxis base es:

\codigofuente{TeX}{Sintaxis base de ftgen}{codigo/ftgenSyntax}

Siendo \textbf{gir} una tabla de al menos 100 posiciones, \textbf{ifn} el número de la tabla. El resto de valores se corresponden con los argumentos \textbf{p2}, \textbf{p3}, \textbf{p4} y \textbf{p5} del \textbf{f Statement} que explicaremos a continuación.

\section{Filtros}
\section{Filtros de onda}
\section{Filtros}
\subsection{sub}


 

