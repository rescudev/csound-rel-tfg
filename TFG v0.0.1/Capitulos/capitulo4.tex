% !TEX root = ../proyect.tex

\chapter{Conceptos Avanzados}

\section{El opcode ftgen}

el opcode \textbf{ftgen} es muy recurrente. Nos sirve para generar una tabla de scores de entre nuestros instrumentos.

Su sintaxis base es:

\codigofuente{TeX}{Sintaxis base de ftgen}{codigo/ftgenSyntax}

Siendo \textbf{gir} una tabla de al menos 100 posiciones, \textbf{ifn} el número de la tabla. El resto de valores se corresponden con los argumentos \textbf{p2}, \textbf{p3}, \textbf{p4} y \textbf{p5} del \textbf{f Statement} que explicaremos a continuación.

\subsection{El f statement}

Coloca valores en una tabla de scores para usar nuestros instrumentos. Está implícita en \textbf{ftgen}.

Esta es la sintaxis base:

\codigofuente{TeX}{Sintaxis base de ftgen}{codigo/fStateSyntax}

Siendo \textbf{p1} el número de tabla, \textbf{p2} tiempo de activación para generar datos, \textbf{p3} tamaño de la tabla, \textbf{p4} nombre de la rutina \textbf{GEN} de generación de datos y \textbf{p5 ... PMAX} que dependerán del \textbf{GEN} que estemos usando.

Veamos un ejemplo en el que rellenamos una tabla (f1) de tamaño 5 con ceros: \textbf{( f1 0 5 2 0\ )}

\section{Delay}
http://write.flossmanuals.net/csound/d-delay-and-feedback/


\section{Modulación de frecuencia}
http://write.flossmanuals.net/csound/d-frequency-modulation/
\section{AM RM WAVESHAPING}
http://write.flossmanuals.net/csound/f-am-rm-waveshaping/
\section{AMPLITUDE AND RING MODULATION}
http://write.flossmanuals.net/csound/c-amplitude-and-ring-modulation/
\section{Filtros de onda}
http://write.flossmanuals.net/csound/c-filters/
\section{Reverberación}
http://write.flossmanuals.net/csound/e-reverberation/

\subsection{sub}


 

