% !TEX root = ../proyect.tex

\chapter{Haciendo Música en Directo}

Este capítulo trata de dar una introducción al ``live coding''\ en Csound. Para ello se enplicarán algunos de sus fundamentos y se usará la página web \url{https://live.csound.com/} (Programada y mantenida por Steven Yi, desarrollador de Csound y blue), de ahora en adelante referida como \textbf{Live Csound}, para ejemplificar los conceptos.

\textbf{Live Csound} usa la librería \textbf{livecode.orc} de Csound y puede abrirse en cualquier navegador commplatible con \textbf{Javascript}, \textbf{WebAudio} y \textbf{WebAssembly}. Cualquiera de los navegadores modernos más usados (Chrome, Firefox, Safari, etc..) tiene estas capacidades. Es además una \textbf{PWA (Progressive Web Application)}\footnote{Revisar el anexo ``Cómo instalar un PWA en Chrome''} por lo que podemos instalar una versión offline de la misma en nuestro equipo y ejecutarla mediante nuestro navegador aunque no dispongamos de conexión a internet.

\section{Live Csound}

\subsection{Atajos de teclado}

Estos son los atajos de teclado más útiles al usar la aplicación \textbf{Live Csound}, es combeniente tenerlos en mente para hacer más dinámico nuestro ``Live Coding''.

\begin{itemize}
 \item \textbf{ctrl-e}: Evalúa y ejecuta el código seleccionado.
 \item \textbf{ctrl-enter}:	Idéntico a ctrl-e.
 \item \textbf{ctrl-shift-enter}:  Evalúa el código seleccionado y lo ejecuta tras un compás de 4/4.
 \item \textbf{ctrl-h}:	Genera una plantilla de código del opcode hexplay().
 \item \textbf{ctrl-j}:	Genera una plantilla de código del opcode euclidplay().
 \item \textbf{ctrl-;}:	Convierte en comentarios las líneas seleccionadas o los descomenta si ya lo eran.
 \item \textbf{ctrl-alt-c}:	Idéntico a ctrl-;.
\end{itemize}

\section{Los principales opcodes}

La biblioteca \textbf{livecode.orc} usada por \textbf{Live Csound} aporta los opcodes \textbf{hexbeat()} y \textbf{hexplay()} que nos servirán generar los sonidos de forma rítmica mediante valores hexadecimales.

\subsection{schedule()}

El opcode \textbf{schedule()} nos servirá para añadir un nuevo score event, es decir, para hacer uso de alguno de sus instrumentos. Su sintaxis siendo la misma que al escribir un score event \textbf{schedule(p1, p2, p3, p4, p5)}. Siendo \textbf{p1} el identificador, \textbf{p2} el instante de inicio, \textbf{p3} la duración, \textbf{p4} la frecuencia del sonido y \textbf{p5} la amplitud en referencia 0dbfs.

\subsection{hexplay()}

Para usar \textbf{hexplay()} es importante haber conocido previamente \textbf{schedule()} puesto que su sintaxis está implícita.  

 \codigofuente{TeX}{Sintaxis base del hexplay()}{codigo/Syntaxhexplay}
 
Siendo \textbf{Spat} un código hexadecimal que marcará la figura rítmica en la que se activa nuestro \textbf{schedule()} implícito. \textbf{Sinstr} es p1, el nombre del instrumento. \textbf{Sinstr} es p3, la duración. \textbf{ifreq} es p4, la frecuencia. Y \textbf{iamp} es p5, la amplitud. No es necesario introducir \textbf{p2} puesto que no habrá un instante inicial particular para cada score event si contamos con un patrón rítmico.

\section{Creando música}

Empecemos por lo tanto con un sencillo ejemplo al que iremos añadiendo dificultad hasta llegar a lo más parecido a una idea musical.

\subsection{Producir sonidos sencillos}

Haremos uso del opcode \textbf{schedule()} del cual ya hemos aprendido la sintaxis.
Vamos a la página \url{https://live.csound.com/}, seleccionamos todo (ctrl-a) y borramos todo el código que aparece como demos dejando la página en blanco. Después de eso podemos escribir la línea de código:

 \codigofuente{TeX}{Ejemplo de schedule()}{codigo/scheduleEx}
 
 Si seleccionamos esta línea de código y pulsamos \textbf{ctrl-e} haremos sonar el instrumento \textbf{``sub1''} desde el instante \textbf{0}, durante \textbf{2} segundos, a una frecuencia de \textbf{440HZ(La)}, y a un volumen relativo de \textbf{0.5} respecto a 0dbfs que va de 0.0 a 1.0.
 
 Y así de sencillo es empezar a producir notas en \textbf{Live Csound}. Por supuesto se recomienda empezar a experimentar con los valores del \textbf{schedule()}, incluido el instrumento para el que existen esta serie de opciones predeterminadas:
 
 \begin{itemize}
 \item \textbf{Instrumentos tonales}: Sub1, Sub2, Sub3, Sub4, Sub5, SynBrass, Plk, Bass, VoxHumana, FM1, Noi, Wobble.
 \item \textbf{Percusión}: Clap, BD, SD, OHH, CHH, HiTom, MidTom, LowTom, Cymbal, Rimshot, Claves, Cowbell, Maraca, HiConga, MidConga, LowConga.
\end{itemize}
 
\subsection{Creando un instrumento}

No queremos vernos limitados a la librería predeterminada de instrumentos de \textbf{Live Csound} así que vamos a aprender a definir nuestro propio instrumento. Esto sigue siendo el lenguaje de programación Csound así que definir un instrumento no es algo nuevo para nosotros. Tratemos de verlo de forma sencilla:

 \codigofuente{TeX}{Un instrumento funcional}{codigo/LiveInstr}
 
En primer lugar vamos a definir un bloque de instrumento llamado \textbf(Add). En \textbf(Add) vamos a definir dos ondas sonoras mediante el opcode \textbf{oscili(xamp, xcps)} donde \textbf{xamp} es la amplitud de onda y \textbf{xcps} su frecuencia.La primerá se llamará |textbf{alfo} y le entregaremos como parámetros de entrada dos valores aleatorios mediante el opcode \textbf{random(min, max)}. 

La segunda onda será \textbf{asig} cuyos valores de entrada dependerán de \textbf{alfo} y de otras operaciones realizadas a la onda bajo un criterio musical (Se invita a experimentar realizando diferentes oprecaiones para encontrar la síntesis de sonido que más de adecúe al gusto propio).

Una vez consigamos el sonido deseado vamos a crear el instrumento \textbf{Note}, en el que generando valores aleatorios de frecuencia, ejecutaremos el instrumento \textbf{Add} siempre en una misma tonalidad musical gracias al opcode \textbf{in\_ scale(ioct, idegree)} que genera una frecuencia válida de entre las notas pertenecientes a la octava \textbf{ioct}.

Al ejecutar la línea \textbf{(schedule("Note", 0, 0))} se generará una nota con las características antes descritas y que sonará durante 20 segundos mientras hace ``fadeout'' (su volumen se reduce progresivamente) debido a que en la línea 5 del código empezamos a reducir el valor de la onda de manera exponencial.

Ya tenemos nuestro instrumento listo, veamos cómo sacarle provecho.

\subsection{Figuras rítmicas como valor hexadecimal}

Vamos a hacer un uso básico del opcode \textbf{hexplay()} para usar el instrumento que hemos creado. 

Para usar \textbf{hexplay()} tendremos que pensar en generar figuras rítmicas mediante valores hexadecimales, teniendo en mente que al pasar el valor hexadecimal a binario cada 1 será un tick en el que se ejecutará nuestro \textbf{schedule()} y cada 0 será silencio.

Tabla de conversión de valores hexadecimales:
\begin{center}
 \begin{tabular}{||c c c c c c c c c c c||} 
 \hline
 HEX & 0 & 1 & 2 & 3 & 4 & 5 & 6 & 7 & 8 & 9 \\
 \hline
 BIN & 0000 & 0001 & 0010 & 0011 & 0100 & 0101 & 0110 & 0111 & 1000 & 1001 \\ 
 \hline
 DEC & 00 & 01 & 02 & 03 & 04 & 05 & 06 & 07 & 08 & 09\\
 \hline
\end{tabular}
\end{center}

\begin{center}
 \begin{tabular}{||c c c c c c c||} 
 \hline
 HEX & A & B & C & D & E & F \\
 \hline
 BIN & 1010 & 1011 & 1100 & 1101 & 1110 & 1111\\ 
 \hline
 DEC & 10 & 11 & 12 & 13 & 14 & 15 \\
 \hline
\end{tabular}
\end{center}
\pagebreak
\codigofuente{TeX}{Uso del hexadecimal para las figuras rítmicas}{codigo/HexRitmoEx}

Tengamos el código de la figura anterior como un ejemplo para entender el funcionamiento del código hexadecimal como atributo de entrada que marca las figuras de ritmo. Prestemos especial atención por tanto a las líneas 5, 10, 15 y 20 del código.

Cuando como argumento \textbf{Spat} del opcode \textbf{hexplay()} introducimos por ejemplo ``1010''\ en valor hexadecimal podemos pensar en ello como un número binario agrupado en cuatro valores, es decir: \textbf{HEX(1010) = BIN(0001 0000 0001 0000)}. 

Cuando marcamos un tempo (en este ejemplo 100) en nuestra pieza y queda el pulso determinado, diremos que cada uno de estos valores hexadecimales representará un pulso. Dicho de otra manera, si ejecutamos únicamente el \textbf{hexplay()} de las líneas 4 a 8 del código, el instrumento \textbf{BD} sonará una vez al final del pulso uno y otra vez al final de pulso tres. Esto se repetirá ad infinitum.

El \textbf{hexplay()} de las líneas 10 a 13 hará sonar el instrumento \textbf{Clap} al final de los pulsos segundo y cuarto.

Cada uno de los pulsos puede dividirse en cuatro partes de igual duración, una por cada bit de los cuatro usados para representar un valor hexadecimal.

El \textbf{hexplay()} de las líneas 15 a 18 hará sonar el instrumento \textbf{LowTom} en la tercera parte de cuarto pulso.

Y por último, el \textbf{hexplay()} de las líneas 20 a 23 hará sonar el instrumento \textbf{Maraca} cuatro veces durante el primer pulso y cuatro veces durante el tercer pulso.

\textbf{hexplay()} acepta cadenas hexadecimales más cortas y más largas que cuatro dígitos como las mostradas en los ejemplos, es por ello que se recomienda encarecidamente abrir \textbf{Live Csound} y empezar a experimentar por uno mismo con los \textbf{hexplay()} para empezar a compreder su concepto rítmico.
\pagebreak
\subsection{Ideas musicales con nuestro instrumento}

Una vez introducido el concepto de figuras rítmicas hexadecimales podemos volver a nuestro instrumento casero y usarlo de la siguiente manera por ejemplo:

\codigofuente{TeX}{Usando hexplay() con nuestro instrumento}{codigo/HexConInstr}

Se ejecutará una vez por cada una de las cuatro parte del cuarto pulso. Al generar, como explicamos antes, una nota con frecuencia aleatoria cada vez; estaremos fabricando acordes arpegiados aleatorios con nuestro instrumento de forma automática cada cuarto pulso del tempo que hayamos elegido.

 

