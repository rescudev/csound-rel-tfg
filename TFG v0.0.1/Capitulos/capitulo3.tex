% !TEX root = ../proyect.tex

\chapter{Profundizando en los conceptos básicos}

Profundicemos en algunos conceptos que crean normalmente confusión entre los usuarios de Csound.

\section{Las diferencias entre variables i-rate y k-rate}

Es común confundir el uso de variables \textbf{i-rate} con el de variables \textbf{k-rate}. Vamos a resolver algunas dudas y a exponer ejemplos de casos en los que usar normalmente cada tipo.

En primer lugar  debemos entender que, como en cualquier lenguaje al uso, las variables en Csound se inicializan al comenzar la ejecuación. La diferencia radical que podemos encontrar a partir de este momento entre variables \textbf{i-rate} y \textbf{k-rate} es que las \textbf{i-rate} van a quedarse con este valor de inicialización. Esto es fácil de entender pero, ¿Qué pasa entonces con las variables \textbf{k-rate}?\bigskip

La duración de un \textbf{k-cycle}, es decir, el tiempo (que puede medirse en cantidad de samples) que pasa desde la útima vez que se refrescaron los valores de las variables \textbf{k-rate} hasta la siguiente vez que se refrescan.

Este valor, \textbf{k-cycle}, es dinámico. Depende de la cantidad de variables tipo k de nuestro código, del valor que le demos a la palabra reservada \textbf{ksmps} y del sample rate seleccionado. Veámoslo con un ejemplos:


\figura{1}{img/3.1-kCycles}{La duración de los K-Cycles}{fig:kCycles}{}

Para el gráfico anterior hemos seleccionado un sr (sample rate) de 44100 y un Block Size (cuyo equivalente en Csound sería la palabra reservada \textbf{ksmps} que mencionamos antes) con valor 10.

Sabemos que el tiempo que tarda en realizarse una muestra o sample es un segundo dividido por el número total de tomas, es decir:\( 1/44100\) que da 0.0000227s.
Y sabemos también que el \textbf{k-cycle} dura 10 samples (también podemos llamarlos ticks), es decir: \( 0.0000227*10= 0.000227s \).

Por lo tanto, en un código con la configuración anterior y un valor de \textbf{ksmps = 10} podremos decir que:

\begin{itemize}
 \item Las variables \textbf{i-rate} determinarán su valor en la inicialización y lo mantendrán durante el tiempo total de ejecución del código.
 \item Las variables \textbf{k-rate} refrescarán su valor cada 10 samples o ticks (que podemos llamar bloque de control), en este caso, se refrescarán cada 0.000227 segundos.
 \item Las variables \textbf{a-rate} refrescarán su valor un vez por cada sample, siendo el ratio de refresco determinado en su totalidad por la frecuencia escogida como sr. En este ejemplo renuevan su valor cada 0.0000227 segundos.
\end{itemize}

Tener un conocimiento suficiente acerca del funcionamiento de las variables \textbf{k-rate} puede hacer que nuestro código acabe siendo mucho más eficiente y optimizado pero como nota general puede bastarnos con recordar que: Las variables \textbf{i-rate} deben usarse cuando sepamos que algo debe ser hecho una única vez y de manera puntual, y las variables \textbf{k-rate} deben usarse cuando necesitemos que algo se haga continuamente pero sepamos que ese algo no necesita ser hecho cada vez que se realiza un sample.

\section{Las f-variables,  w-variables y S-variables}

Existen dos tipos de variables en csound que son algo más especiales que las vistas hasta ahora:

\begin{itemize}
 \item \textbf{Las f-variables}: Son variables usadas por algunos opcodes (los que empiezan por \textbf{pvs}), y se usan principalmente para la realización de \textsl{Transformadas rápidas de Fourier}. Su ratio de refresco es el mismo que para las variables \textbf{k-rate} pero su valor depende de algunos parámetros  de las transformadas que hemos mencionado.
 \item \textbf{Las w-variables}: Podemos encontrar el uso de estas variables en algunos opcodes antiguos aunque su uso es ya prácticamente por razones hereditarios por lo que no profundizaremos en ellas.
 \item \textbf{Las S-variables}: Son variables de tipo \textbf{String}, serán necesarias para usar algunos opcodes cuyo resultado sea una variable de este tipo.
\end{itemize}

\section{El ámbito global y local de las variables}

Las variables contenidas en el código de un instrumento son generalmente de ámbito local, es decir, podría crear una variable con el mismo nombre dentro de todos mis instrumento sin que exista ningún tipo de conflicto.

Las variables globales sin embargo deben ser únicas cada vez que escribamos un valor sobre ellas, su valor cambiará para cualquier futura lectura. Para hacer que una variable sea de tipo global debemos hacer que \textbf{g} sea la primera letra del nombre de esa variable. Algunos ejemplo de nombre de variable global serían: \textbf{gaGlobal}, \textbf{giConstante} o \textbf{gkResultado}.

Una alternativa al uso de las variables globales es el uso del opcode \textbf{chnget} para realizar conexiones de canal entre variables. Es el método usado en la etiqueta \textless Cabbage\textgreater para relacionar los widgets con el resto del código sin hacer uso de variables globales.

\section{Las estructuras de control}

En Csound como en la mayoría de lenguajes existen la estructuras de control: Sentencias \textbf{if-else}, bucles \textbf{while/until} y los llamados \textbf{timouts}. Vamos a explicarlas centrándonos en la peculiaridades de uso respecto al lenguaje.

\subsection{Sentencias if-else}

La forma más común de este tipo de sentencia en Csound es \textbf{If - then - [elseif - then -] else}:

\codigofuente{TeX}{Sintaxis base de la sentencia if-else}{codigo/ifejemplo}

Lo único a destacar sería que la palabra \textbf{then} debe estar en la misma línea de código que la palabra \textbf{if}, pero no ahondaremos más en este tipo de sentencia al tratarse de nociones básicas de la programación.

Csound permite también la sintaxis de lenguaje descriptivo \textbf{(a v b ? x : y)}: De ser verdadera la condición \textbf{a}, el valor devuelto es \textbf{x}. De ser falsa (y por tanto ser cierta \textbf{b}) el valor devuelto es \textbf{y}. Un ejemplo práctico de uso sería: \textbf{kRes = (kVar \textless 1 ? 0 : 1);}. Si kVar es menor que uno se devuelve 0, se no ser así se devuelve 1.

\subsection{Bucles While/Until}

\codigofuente{TeX}{Sintaxis base de los bucles while-until}{codigo/whileejemplo}

Estos bucles funcionan de forma análoga, la única diferencia entre ellos es que el bucle \textbf{while} se seguirá ejecutando siempre y cuando la condicion sea verdadera y el bucle \textbf{until} se seguirá ejecutando siempre y cuando la condición sea falsa.

\subsection{El timout}

El \textbf{timout} es un opcode para generar bucles de una duración determinada. 

\codigofuente{TeX}{Sintaxis base del timout}{codigo/timoutSyntax} 

En primer lugar \textbf{first\_ label} y \textbf{second\_ label} son etiquetas de referencia a las que podemos saltar desde otras partes del código. El opcode \textbf{(timout    istart, idur, second\_ label)} tiene tres parámetros de entrada: \textbf{istart} el instante de inicio, \textbf{idur} la duración de timout y \textbf{second\_ label} el nombre de la etiqueta de la parte del código a la que queremos saltar, en este caso por lógica la segunda.El segundo opcode necesario para el funcionamiento es \textbf{(reinit	first\_ label)} que da la orden directa de saltar a la parte del código referida por \textbf{first\_ label}.

Entendamos el uso de \textbf{timout} con un ejemplo práctico:

\codigofuente{TeX}{Ejemplo real de uso del timout}{codigo/timoutReal}

Hemos definido una variable \textbf{idur} a la que damos un valor aleatorio entre (0,5 y 3) mediante el opcode \textbf{random}. Acto seguido usamos el opcode \textbf{timout} para saltar desde el instante \textbf{0}, durante ese \textbf{valor aleatorio de segundos}, a la etiqueta \textbf{play} donde se ejecuta el opcode \textbf{poscil} que genera una onda de sonido. Una vez acabado ese periodo de tiempo se ejecuta el opcode \textbf{reinit} que nos hace saltar a la etiqueta \textbf{loop} y vuelta a empezar.

\section{Los Arrays de datos}

Al igual que en cualquier lenguaje con cierto nivel de complejidad, en Csound existen los \textbf{Arrays} o vectores de datos. Veamos las peculiaridades del lenguaje en el uso de este tipo de estructuras.

\subsection{Propiedades de los Arrays}

En Csound podemos pensar en cinco propiedades características al definir un Array:

\begin{itemize}
 \item \textbf{Dimensiones}: Los elementos de un array pueden leerse mediante la sintaxis \textbf{kArr[i]} siendo \textbf{i} la posición del elemento al que queremos acceder en un array unidimensional. De igual manera podemos acceder a los elementos de un array de dos dimensiones con la sintaxis \textbf{kArr[i][j]}. De forma análoga para arrays tridimensionales.
 \item \textbf{i- o k-Rate}: Los arrays son variables y como tal pueden definirse como variable i-rate o k-rate.
 \item \textbf{Local o Global}: De la misma manera podemos hacer de nuestro array una variabe global añadiendo la letra \textbf{g} al principio del nombre.
 \item \textbf{Arrays de Strings}: Los arrays de Csound pueden contener variables String además de número por lo que podemos clasificarlos también de esta manera.
 \item \textbf{Arrays de señales digitales}: Por último podemos conectar canales y salidas de opcodes a las posiciones de un array para facilitar el trabajo con las señales de audio.
\end{itemize}
\pagebreak

\subsection{Opcodes útiles}

Estos son algunos de los opcodes que podemos usar al trabajar con arrays:

\begin{itemize}
 \item \textbf{init}: El opcode que usamos para inicializar un array, para su sintexis únicamente necesitamos aportar el tamaño de cada una de las dimensiones del array que estamos definiendo. Por ejemplo: \textbf{kArr[]   init 5} para un array unidimensional de longitud 5.
 \item \textbf{fillarray}: Para añadir una serie de valores a nuestro array. Veamos un ejemplo: \textbf{kArr[] fillarray 1, 2, 3, 4, 5} que añade los valores 1, 2, 3, 4 y 5 al array.
 \item \textbf{genarray}: Genera un array al que se le añaden los valores comprendidos entre los valores de entrada del opcode. Un ejemplo de uso: \textbf{kArr[] genarray   1, 5} que crea un array al que se le añaden los valores 1, 2, 3, 4 y 5.
 \item \textbf{lenarray}: Devuelve el tamaño actual de un array. \textbf{kTam  lenarray  kArr} devolvería el tamaño de kArr. 
 \item \textbf{slicearray}: Nos sirve para generar un subarray desde un índice inicial hasta un índice final del array original introducido como parámetro de entrada en el opcode. Su sintaxis base es: \textbf{kSlice[] slicearray kArr, iStart, iEnd}. Y un ejemplo de uso sería \textbf{kSub[]  slicearray kArr, 0, 2} que para el array \textbf{kArr=[4,3,2,1,0]} devolvería como resultado \textbf{kSub=[4,3,2]}
 \item \textbf{minarray}: Este opcode devuelve el valor más pequeño de todo el array y de manera opcional su índice si especificamos una variable para guardarla. Su sintaxis base  es: \textbf{kMin [,kMinIndx] minarray kArr}.
 \item \textbf{maxarray}: Este opcode devuelve el mayor valor de todo el array y de manera opcional su índice si especificamos una variable para guardarla. Su sintaxis base  es: \textbf{kMin [,kMinIndx] minarray kArr}.
 \item \textbf{sumarray}: Devuelve la suma de todos los valores del array numérico. Su sintaxis es \textbf{kSum sumarray kArr} y siendo \textbf{kArr=[1,1,1,1]} resultaría \textbf{kSum = 3}.
 \item \textbf{scalearray}: Escala los valores de un array en referencia a un valor mínimo y a un valor máximo. Veamos un ejemplo de uso: 
  
\codigofuente{TeX}{Uso del scalearray}{codigo/scaleArray}  

Tenemos el array \textbf{kArr=[1,3,9,5,6]}, donde el valor más pequeño es 1 y el valor más grande es 9. Cuando hacemos uso del opcode \textbf{(scalearray kArr 1,3)} estaremos escalando con las siguientes referencias relativas: \textbf{1 $\leftrightarrow$ 1 (kmin)} y \textbf{9 $\leftrightarrow$ 3 (kmax)}. De la misma manera el resto de valores de \textbf{kArr} quedarán en referencia también a kmin y kmax. El resultado para \textbf{(scalearray kArr 1,3)} con \textbf{kArr=[1,3,9,5,6]} sería  \textbf{kArr=[1, 1.5, 3, 2, 2.25]}.

 \item \textbf{maparray}: Aplica un opcode en formato función a cada uno de los valores del array. Su sintaxis base es: \textbf{(kRes  maparray kArr, ``fun")}. Siendo \textbf{fun} la función que queremos usar de entre las siguientes opciones: \textbf{abs, ceil, exp, floor, frac, int, log, log10, round, sqrt}. Por ejemplo, la operación \textbf{(kRes  maparray kArr, sqrt)} aplica la función \textbf{sqrt()} a cada elemento del array \textbf{kArr} y almacena los resultados en \textbf{kRes}.
  
\end{itemize}

\subsection{Operaciones con arrays}

Se pueden usar los cuatro operadores básicos (\textbf{+, -, *, /}) para sumar, restar, multiplicar y dividir arrays. Si el operador se aplica entre el array y un número, la operación se realiza a todos los elementos del array, por lo que guardar los resultados en un nuevo array es una buena práctica. Por ejemplo si tenemos el array \textbf{kArr=[1,1,1]} y realizamos la operación \textbf{kSuma = Karr + 1} obtendremos \textbf{kSuma=[2,2,2]}.

Por otra parte, si realizamos la operación entre dos arrays del mismo tamaño, el cálculo tendrá en cuenta cada pareja de valores de la misma posición de cada array. Si tenemos el array \textbf{kArr1=[10,10,10]} y el array \textbf{kArr2=[1,2,3]} y realizamos la operación \textbf{kArr3 = kArr1 + kArr2} obtenemos \textbf{kArr3=[11,12,13]}.

\section{Funciones de entrada/salida}

Muchos de los opcodes de Csound pueden expresarse mediante la sintaxis funcional que se usa en tantos otros lenguajes: \textbf{fun(arg1, agr2...)}. Hablemos de algunos de los opcodes que más se usan en su formato funcional.

\begin{itemize}
 \item \textbf{abs}: Devuelve el valor absoluto del parámetro de entrada \textbf{arg}. 
 
 Sintaxis: \textbf{abs(arg)}
 
 Ejemplo: \textbf{abs(-2) = 2}
 \item \textbf{ceil}: Devuelve el menor número entero posible que sea mayor que \textbf{arg}.
 
 Sintaxis: \textbf{ceil(arg)}
 
 Ejemplo: \textbf{ceil(0.999) = 1}
 \item \textbf{exp}: Devuelve el número \textbf{e (2,718...)} elevado a la potencia de \textbf{arg}.
 
 Sintaxis: \textbf{exp(arg)}
 
 Ejemplo: \textbf{exp(2) = 7.389}
 \item \textbf{frac}: Devuelve la parte fraccionaria de \textbf{arg}.
 
 Sintaxis: \textbf{frac(arg)}
 
 Ejemplo: \textbf{frac(2.1234) = 0.1234}
 \item \textbf{int}: Devuelve la parte entera de \textbf{arg}.
 
 Sintaxis: \textbf{int(arg)}
 
 Ejemplo: \textbf{int(2.1234) = 2}
 \item \textbf{log}: Devuelve el logaritmo natural de \textbf{arg}.
 
 Sintaxis: \textbf{log(arg)}
 
 Ejemplo: \textbf{log(8) = 2.079}
 \item \textbf{sqrt}: Devuelve la raíz cuadrada de \textbf{arg}.
 Sintaxis: \textbf{sqrt(arg)}
 
 Ejemplo: \textbf{sqrt(25) = 5}
\end{itemize}

\section{Creando un opcode (UDOs)}

En Csound, un opcode creado por un usuario recibe el nombre de \textbf{UDO (User created opcode)}. Definir un \textbf{UDO} es simplemente escribir parte del código que incluiríamos en un instrumento, dentro de un bloque especial para luego poder usarlo a efectos prácticos de la misma manera que usamos cualquier opcode.

La sintaxis base para definir un UDO es:

\codigofuente{TeX}{Sintaxis base de un UDO}{codigo/UDOSyntax} 

Vamos por partes, las palabras \textbf{opcode} y \textbf{endop} marcan el inicio y el final del bloque de código en el que se define nuestro UDO. En la línea 1 de la figura de código anterior observamos: \textbf{name} que será el nombre de nuestro UDO, \textbf{outtypes} donde marcaremos el tipo de variable de las variables de salida del UDO y \textbf{intypes} donde marcaremos el tipo de variable de las variables de entrada. La sintaxis para usar \textbf{outtypes} e \textbf{inttypes} es simplemente un string con una letra por cada variable de entrada o salida especificando su tipo.

Por ejemplo, esta sería la cabecera del bloque de código de un UDO con dos variables de salida, una tipo \textbf{i-rate} y otra \textbf{tipo k-rate}; y una variable de entrada \textbf{k-rate}: \textbf{(opcode miOpcode, ik, k)}

Los opcodes \textbf{xin} y \textbf{xout} nos sirven respectivamente para recoger las variables de entrada, dándoles un nombre identificativo para usarlo en el resto del UDO, y para definir cuál o cuáles serán las variables de salida.

Por último, \textbf{inName} y \textbf{outName} representan los nombres de esas variables de entrada y salida que además deben respetar la nomenclatura de uso de tipos en Csound empezando por a, i, k, etc... Según corresponda en la defición de la primera línea del bloque.

Vamos a crear un ejemplo de UDO para realizar ecuaciones de segundo grado: \( x = \frac {-b \pm \sqrt {b^2 - 4ac}}{2a} \)

\codigofuente{TeX}{UDO para ecuaciones de segundo grado}{codigo/UDOSegundoGrado}

Y para usar nuestro opcode en un instrumento:

\codigofuente{TeX}{Uso del opcode SegundoGrado en un instrumento}{codigo/InstSegundoGrado}

\pagebreak


\section{Las macros}

En Csound pueden usarse \textbf{macros} para tener una o varias líneas de código a mano y sustituirlas siempre que queramos con un nombre de referencia a elegir. Normalmente las usaremos cuando tengamos previsto repetir una serie de líneas de código a lo largo de nuestro programa o cuando hagamos referencia a algún nombre de variable o archivo para que, en caso de querer cambiar su nombre o valor, únicamente necesitemos hacerlo en la definición de la \textbf{macro} y no cada vez que aparezca en nuestro código.

\subsection{Sintaxis de las macros}

Su sintaxis es sencilla: \textbf{(\#define NAME \# Código o texto a sustituir \#)}

Siendo \textbf{define} la palabra reservada para empezar a definir una \textbf{macro}, \textbf{NAME} el nombre con el que más tarde haremos referencia a la macro, y la parte contenida entre el segundo y tercer corchete el bloque de código o texto que queda sustituido y que aparecerá repetido a efectos de compilación cada vez que usemos la \textbf{macro}.

Para hacer referencia a la macro en otros lugares de código bastará con escribir su nombre precedido del símbolo \textbf{\$}

Veamos un ejemplo de uso:

\codigofuente{TeX}{Ejemplo de una macro simple}{codigo/macroSimple}

Como vemos hemos definido la macro \textbf{ejemploMacro} que contiene una única línea de código para imprimir por consola el texto ``Soy la macro''. Más tarde usamos esta macro 3 veces en nuestro instrumento por lo que a efectos prácticos el código \textbf{prints ``Soy la macro "} se ejecutará un total de 3 veces.

\subsection{Macros con parámetros de entrada}

Podemos también definir macros con parámetros de entrada de la siguiente manera: 

\textbf{(\#define NAME(Arg1'\ Arg2'\ Arg3...) \# Código o texto a sustituir \#)}

Simplemente incluiremos entre paréntesis la serie de argumentos que usará nuestra macro. La separación entre nombre y nombre de argumento la marcaremos mediante el carácter (\textbf{'}). Para usar los argumentos dentro del bloque de código de la macro les haremos referencia precendiéndolos del símbolo \textbf{\$}.

Veamos un ejemplo de uso:

\codigofuente{TeX}{Ejemplo de una macro con argumentos}{codigo/macroArgs}

\subsection{Macros en \textless CsScore\textgreater}

Hasta ahora hemos estado definiendo el código de nuestras macros en \textless CsInstruments\textgreater pero también podemos usarlas en \textless CsScore\textgreater. De esta manera pueden sernos útiles para realizar pruebas de testeo ordenadas y controladas o, como hicimos anteriormente, para tener un código más reutilizable y optimizado.

La sintaxis es idéntica al caso anterior así que veamos un ejemplo común de uso:

\codigofuente{TeX}{Ejemplo de una macro en \textless CsScore\textgreater}{codigo/macroInScore}

Imaginemos para el ejemplo un instrumento que genera una determinada nota y que usa p1, p2, p3 y p4; siendo p4 un parámetro que determina la frecuencia del tono de la nota de salida del instrumento. En nuestra macro realizamos tres \textbf{Event scores} (líneas 3, 4 y 5) usando ese instrumento, cuyos p2 (momento de inicio) y p4 (tono de la nota de salida) se ven modificados por \textbf{Start} y \textbf{Trans} que son los parámetros de entrada de nuestra macro. De esta manera podemos realizar esos tres \textbf{Score events} tantas veces como queramos y únicamente introduciendo los parámetros que consideramos importantes al definir la macro.
