\chapter{Anexos}\label{anexos}

\section{Cómo instalar un PWA en Chrome}

En primer lugar visitamos la página web del PWA y hacemos click es la pestaña de opciones. Acto seguido vamos a la opción: \textbf{Más herramientas\textgreater Crear acceso directo...}.

\figura{0.8}{img/A.1-PWAinst}{Crear acceso directo a un PWA}{fig:PWA}{}

Chrome nos preguntará por confirmación para realizar la instalación. Bastará con pulsar en ``Aceptar'' y se creará un acceso directo en nuestro escritorio que podremos ejecutar siempre que queramos.
\pagebreak
\section{Código de la sección ``Un ejemplo más completo'' del capítulo 6}

\codigofuente{TeX}{Ejemplo 2 Csound en Java, programa completo}{codigo/CsoundEnJavaComp}

