% !TEX root = ../proyect.tex

\chapter{La Sintaxis del Lenguaje}

\section{Partes}\label{sec:partes}

Todas las memorias de Trabajo fin de Grado deberán constar de las siguientes partes 
\begin{itemize}
 \item Portada (seg\'un formato oficial). No debe incluir n\'umero de p\'agina. Debe incluir:
 \begin{itemize}
 \item Sello de la universidad de Sevilla a dos tintas
 \item ESCUELA T\'ECNICA SUPERIOR DE INGENIER\'IA INFORM\'ATICA
 \item Trabajo fin de Grado
 \item \emph{Denominaci\'on del Grado}
 \item Realizado por: \emph{Nombre y apellidos del estudiante}
 \item Dirigido por: \emph{Nombre y apellidos del tutor o tutores}
 \item Departamento \emph{Nombre del departamento en el que se lee el TfG}
 \item Sevilla, \emph{Mes y año de la convocatoria de entrega}
 \end{itemize}
  \item Preliminares: las p\'aginas no se numeran o se numeran con n\'umeros romanos.
 \begin{itemize} 
\item Resumen en castellano (m\'aximo una hoja)
 \item Abstract (resumen en ingl\'es, obligatorio para el caso de las memorias escritas en ingl\'es, opcional
 para las escritas en castellano)
 \item Agradecimientos (opcional)
 \item \'Indice general (contenido de la memoria, con menci\'on de las partes en que está dividida)
 \item \'Indice de figuras (opcional)
 \item \'Indice de cuadros (mal llamados en general, tablas) (opcional)
 \item \'Indice de c\'odigo o algoritmos (opcional). 
 \end{itemize}
 \item Cuerpo de la memoria, dividida en cap{\'\i}tulos. El contenido de la memoria ha de incluir 
los elementos característicos de un proyecto de ingeniería o de un estudio o trabajo en el ámbito de una investigación, 
los cuales, en el sentido más amplio, son:
 \begin{itemize}
 \item Definición de objetivos.
\item Análisis de antecedentes y aportación realizada.
\item Análisis temporal y de costes de desarrollo.
\item Análisis de requisitos, diseño e implementación.
\item Manual de usuario, en su caso.
\item Pruebas.
\item Comparación con otras alternativas.
\item Conclusiones y desarrollos futuros
 \end{itemize}
Estos puntos podrán ser ajustados y modificados en función de la naturaleza del proyecto realizado.
 \item Bibliograf{\'\i}a: se deben documentar las fuentes bibliográficas utilizadas en el formato APA 2009.
 \item  Índice alfabético o glosario (Lista ordenada de los conceptos, los nombres propios, etc.; 
 que aparecen en la memoria, con las indicaciones necesarias para su localización)(opcional)
 \item Ap'endices: Si la memoria contiene alg\'un art{\'\i}culo de investigaci\'on o similar, 
 este se incluir\'a en los Ap\'endices. 
 Es necesario, en ese caso, incluir en la página anterior una hoja con la citación bibliográfica. 
 En esta citación, el título del artículo se debe enlazar con la página web de la revista 
 en la que aparecen el resumen o abstract y el acceso al texto completo. 
 \end{itemize}
 La numeraci\'on de las p\'aginas de la bibliograf{\'\i}a y del glosario debe continuar la del cuerpo de la memoria.
 Los ap\'endices pueden llevar su propia numeraci\'on independiente o usar la general del cuerpo de la memoria.


\section{Impresión}\label{sec:impresion} 
La entrega del memoria y en su caso, el dep\'osito en biblioteca, se hacen en formato electr\'onico. Debido a ello, la memoria se presentará \emph{a una cara}. Si se requiriera
por alg\'un motivo la impresi\'on de la misma, se recomienda vivamente preparar la memoria adecuadamente para su impresi\'on. Algunas sugerencias:
\begin{itemize}
 \item Dejar una p\'agina en blanco cuando sea necesario para que los cap{\'\i}tulos comiencen siempre en p\'agina impar (derecha)
 \item Ajustar los m\'argentes para que el exterior sea ligeramente m\'as grande que el interior.
 \item Ajustar las cabeceras y pies de p\'agina (en su caso). Por ejemplo, si el n\'umero de p\'agina ocurriese en un lateral
 de la cabecera o pie, este debe ser siempre el exterior (derecho para las p\'aginas impares, izquierdo para las pares)
\end{itemize}

\section{Plantillas}\label{sec:plantillas} 

Se encuentra disponible una plantilla LaTex en la secci\'on de documentos de la plataforma
de la aplicaci\'on de TfG de la ETSII 
\url{https://tfc.eii.us.es/TfG/}. Esa plantilla ha sido utilizada para preparar este documento.
Se espera en breve disponer de plantillas de ejemplo para las aplicaciones OpenOffice Writer y Office Word.

Nota: Las plantillas se proporcionan como ejemplo, las condiciones obligatorias son las que constan en este procedimiento.

\chapter{Ejemplos}\label{cap2} 
					
\section{Manejo de la bibliograf{\'\i}a}
En esta secci\'on mostramos brevemente ejemplos de referencia a la bibliograf{\'\i}a 
citando un libro~\cite{desousa}, un art{\'\i}culo~\cite{guiatitlesec} 
y una p\'agina web~\cite{informatica}. 

Se recuerda que son campos obligatorios
en todos los {\'\i}tems de la bibliograf{\'\i}a: 
autor(es), t{\'\i}tulo del libro o art{\'\i}culo 
% editorial (en su caso, tambi\'en revista) 
y a{\~n}o de publicaci\'on.
En el caso de p\'aginas web, 
es obligatoria la fecha de la \'ultima consulta. 
En general, la bibliograf{\'\i}a debe ayudar al lector a encontrar f\'acilmente los {\'\i}tems citados.

\section{C\'odigo}

En general se debe evitar incluir c\'odigo o seudoc\'odigo en la memoria. Si fuese preciso, se destacar\'a de forma
que sea f\'acilmente identificable y se indexar\'an los trozos de c\'odigo incluidos. Un ejemplo puede verse
a continuaci\'on.

\codigofuente{TeX}{Código de ejemplo en LaTeX}{codigo/macrotabla}

\section{Im\'agenes}

Este es un ejemplo de inclusi\'on de figura en el texto (v\'ease la figura~\ref{fig:ada}). 

\figura{0.5}{img/AdaLovelace}{Ada Lovelace}{fig:ada}{}

Figuras y cuadros se colocar\'an preferentemente tras el p\'arrafo en el que
son llamados por primera vez. Si no cupieran, se colocar\'an (en orden de preferencia):
\begin{itemize}
 \item Al final de la p\'agina en que se llaman
 \item Al principio de la siguiente p\'agina
 \item Al final del cap{\'\i}tulo
\end{itemize}
siempre respetando el orden de aparici\'on en el texto.

\section{Cuadros (mal llamados Tablas)}

Este es un ejemplo de inclusi\'on de cuadro en el texto. V\'ease el cuadro~\ref{tab:prueba}

\cuadro{|r||c|c|}{Cuadro de prueba}{tab:prueba}{elemento & elemento & elemento \\ elemento & elemento & elemento \\}