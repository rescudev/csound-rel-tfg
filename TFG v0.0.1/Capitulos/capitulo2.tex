% !TEX root = ../proyect.tex

\chapter{La Sintaxis del Lenguaje}

\section{Hello Csound!}\label{sec:hello}

Empecemos por mostrar cómo sería el clásico \textsl{Hello World!} en Csound. Así tendremos un código básico al que hacer referencia a los largo de este capítulo\footnote{Durante el resto del presente capítulo se hará referencia a la figura de código anterior siempre que se hable de líneas concretas de código.}:
\codigofuente{TeX}{Hello World! en Csound}{codigo/helloWorld}

\section{A tener en cuenta}\label{sec:cuenta}

Estas son algunas consideraciones básicas del lenguaje que debemos tener presentes:
\begin{itemize}
 \item Es sensible a mayúsculas/minúsculas.
 \item Usa especificadores de formato al imprimir variables.
 \item Para realizar comentarios en el código usaremos ; al inicio de la línea o usar /* y */ para comentar varias líneas.
 \item Partes del código divididas en etiquetas XML.
\end{itemize}

\section{División por etiquetas}\label{sec:etq}

Csound divide la estructura de su código con etiquetas XML, empezaremos por hablar de las más básicas: \textless CsInstruments\textgreater y \textless CsScore\textgreater. Hasta llegar a etiquetas más exclusivas como \textless Cabbage\textgreater.

\begin{itemize}
 \item \textbf{Etiqueta \textless CsInstruments\textgreater}: En esta etiqueta se incluirán las definiciones de los instrumentos que crearemos. Un poco más adelante trataremos de entender qué es un instrumento en Csound. En nuestro ejemplo la etiqueta \textless CsInstruments\textgreater abarca desde la línea 4 a la línea 12, y podemos observar la definición de un instrumentos entre las líneas 6 y 10.
 \item \textbf{Etiqueta \textless CsScore\textgreater}: Aquí haremos uso práctico de nuestros instrumentos, les diremos cómo ejecutarse y durante cuánto tiempo. En el código del que disponemos, la etiqueta \textless CsScore\textgreater abarca desde la línea 13 a la 15 y tenemos un ejemplo sencillo de ejecución en la línea 14 al que volveremos más adelante.
 \item \textbf{Etiqueta \textless CsOptions\textgreater}: Se inclurán aquí las especificaciones técnicas para interactuar con hardware u otros dispositivos.
 \item \textbf{Etiqueta \textless CsoundSynthesizer\textgreater}: Todo el código, incluidas las etiquetas mencionadas anteriormente, debe estar incluido en \textless CsoundSynthesizer\textgreater. Es la forma que tiene el compilador de saber dónde empieza y dónde acaba el código Csound.
 \item \textbf{Etiqueta especial \textless Cabbage\textgreater}: Es la única etiqueta que no debe estar dentro de \textless CsoundSynthesizer\textgreater puesto que es exlusiva de Cabbage, IDE que usaremos principalmente en este documento. En esta etiqueta incluiremos el código referente a las opciones de personalización de interfaz gráfica de nuestro programa, hablaremos más de ella en secciones referentes al uso del IDE Cabbage.
\end{itemize}

\section{Palabras reservadas}\label{sec:reservadas} 



\section{Plantillas}\label{sec:plantillas} 

Se encuentra disponible una plantilla LaTex en la secci\'on de documentos de la plataforma
de la aplicaci\'on de TfG de la ETSII 
\url{https://tfc.eii.us.es/TfG/}. Esa plantilla ha sido utilizada para preparar este documento.
Se espera en breve disponer de plantillas de ejemplo para las aplicaciones OpenOffice Writer y Office Word.

Nota: Las plantillas se proporcionan como ejemplo, las condiciones obligatorias son las que constan en este procedimiento.

\chapter{Ejemplos}\label{cap2} 
					
\section{Manejo de la bibliograf{\'\i}a}
En esta secci\'on mostramos brevemente ejemplos de referencia a la bibliograf{\'\i}a 
citando un libro~\cite{desousa}, un art{\'\i}culo~\cite{guiatitlesec} 
y una p\'agina web~\cite{informatica}. 

Se recuerda que son campos obligatorios
en todos los {\'\i}tems de la bibliograf{\'\i}a: 
autor(es), t{\'\i}tulo del libro o art{\'\i}culo 
% editorial (en su caso, tambi\'en revista) 
y a{\~n}o de publicaci\'on.
En el caso de p\'aginas web, 
es obligatoria la fecha de la \'ultima consulta. 
En general, la bibliograf{\'\i}a debe ayudar al lector a encontrar f\'acilmente los {\'\i}tems citados.

\section{C\'odigo}

En general se debe evitar incluir c\'odigo o seudoc\'odigo en la memoria. Si fuese preciso, se destacar\'a de forma
que sea f\'acilmente identificable y se indexar\'an los trozos de c\'odigo incluidos. Un ejemplo puede verse
a continuaci\'on.

\codigofuente{TeX}{Código de ejemplo en LaTeX}{codigo/macrotabla}

\section{Im\'agenes}

Este es un ejemplo de inclusi\'on de figura en el texto (v\'ease la figura~\ref{fig:ada}). 

\figura{0.5}{img/AdaLovelace}{Ada Lovelace}{fig:ada}{}

Figuras y cuadros se colocar\'an preferentemente tras el p\'arrafo en el que
son llamados por primera vez. Si no cupieran, se colocar\'an (en orden de preferencia):
\begin{itemize}
 \item Al final de la p\'agina en que se llaman
 \item Al principio de la siguiente p\'agina
 \item Al final del cap{\'\i}tulo
\end{itemize}
siempre respetando el orden de aparici\'on en el texto.

\section{Cuadros (mal llamados Tablas)}

Este es un ejemplo de inclusi\'on de cuadro en el texto. V\'ease el cuadro~\ref{tab:prueba}

\cuadro{|r||c|c|}{Cuadro de prueba}{tab:prueba}{elemento & elemento & elemento \\ elemento & elemento & elemento \\}