% !TEX root = ../proyect.tex

\chapter{La Sintaxis del Lenguaje}

\section{Hello Csound!}\label{sec:hello}

Empecemos por mostrar cómo sería el clásico \textsl{Hello World!} en Csound. Así tendremos un código básico al que hacer referencia a los largo de este capítulo\footnote{Durante el resto del presente capítulo se hará referencia a la figura de código anterior siempre que se hable de líneas concretas de código.}:
\codigofuente{TeX}{Hello World! en Csound}{codigo/helloWorld}

\section{A tener en cuenta}\label{sec:cuenta}

Estas son algunas consideraciones básicas del lenguaje que debemos tener presentes:
\begin{itemize}
 \item Es sensible a mayúsculas/minúsculas.
 \item Usa especificadores de formato al imprimir variables.
 \item Para realizar comentarios en el código usaremos ; al inicio de la línea o usar /* y */ para comentar varias líneas.
 \item Partes del código divididas en etiquetas XML.
\end{itemize}

\section{División por etiquetas}\label{sec:etq}

Csound divide la estructura de su código con etiquetas XML, empezaremos por hablar de las más básicas: \textless CsInstruments\textgreater y \textless CsScore\textgreater. Hasta llegar a etiquetas más exclusivas como \textless Cabbage\textgreater.

\begin{itemize}
 \item \textbf{Etiqueta \textless CsInstruments\textgreater}: En esta etiqueta se incluirán las definiciones de los instrumentos que crearemos. Un poco más adelante trataremos de entender qué es un instrumento en Csound. En nuestro ejemplo la etiqueta \textless CsInstruments\textgreater abarca desde la línea 4 a la línea 12, y podemos observar la definición de un instrumentos entre las líneas 6 y 10.
 \item \textbf{Etiqueta \textless CsScore\textgreater}: Aquí haremos uso práctico de nuestros instrumentos, les diremos cómo ejecutarse y durante cuánto tiempo. En el código del que disponemos, la etiqueta \textless CsScore\textgreater abarca desde la línea 13 a la 15 y tenemos un ejemplo sencillo de ejecución en la línea 14 al que volveremos más adelante.
 \item \textbf{Etiqueta \textless CsOptions\textgreater}: Se inclurán aquí las especificaciones técnicas para interactuar con hardware u otros dispositivos.
 \item \textbf{Etiqueta \textless CsoundSynthesizer\textgreater}: Todo el código, incluidas las etiquetas mencionadas anteriormente, debe estar incluido en \textless CsoundSynthesizer\textgreater. Es la forma que tiene el compilador de saber dónde empieza y dónde acaba el código Csound.
 \item \textbf{Etiqueta especial \textless Cabbage\textgreater}: Es la única etiqueta que no debe estar dentro de \textless CsoundSynthesizer\textgreater puesto que es exlusiva de Cabbage, IDE que usaremos principalmente en este documento. En esta etiqueta incluiremos el código referente a las opciones de personalización de interfaz gráfica de nuestro programa, hablaremos más de ella en secciones referentes al uso del IDE Cabbage.
\end{itemize}

\section{Palabras reservadas}\label{sec:reservadas} 

Las palabras reservadas son variables globales con una funcionalidad especial para Csound como delimitar bloques de código o determinando valores configurables. Pueden inicializarse a un valor determinado en nuestro código que queda posteriormente grabado en el tiempo  de compilación.
Veamos algunas de las palabras reservadas más comunes en Csound:

\begin{itemize}
 \item \textbf{instr/endin}: Las palabras reservadas instr y endin sirven para determinar el comienzo y el final del bloque de código necesario para crear un instrumento en Csound. Podemos ver un ejemplo de uso en las líneas 6 y 10 de la figura de código 2.1.
 Además debemos asignar un nombre o identificador al instrumento acompañando a la palabra reservada instr, en nuestro ejemplo de código le damos el nombre ``1'' \ a nuestro instrumento.
 
 \item \textbf{sr}: Indica el valor del sample rate. Wl valor predeterminado es de 44100Hz, es decir, 44100 veces cada segundo. Normalmente usaremos un valor 44100 o de 48000 según la compatibilidad de la tarjeta de sonido de nuestro equipo.\footnote{Para mayor entendimiento de los conceptos relacionados al sampleo véase el apartado ``El Sampleo y Sample Rate'' del capítulo ``Fundamentos del Sonido''}
 
 \item \textbf{nchnls}: Se trata del número de canales de salida de audio que usamos en nuestro programa. Con nchnls = 1 conseguimos sonido mono, con nchnls = 2 stereo, nchnls = 4 cuadrafónico. (véase figura 2.1)
 \figura{0.8}{img/2.1-Cuadrafonico}{Sonido cuadrafónico}{fig:quadra}{}

 \item \textbf{0dbfs}: Determina el valor relativo que usaremos en el programa como 0 decibelios y a partir del cual aumentaremos o disminuiremos los decibelios de volumen. Por defecto tiene un valor de 32767. \footnote{Para mayor entendimiento del concepto de decibelio, ir a la sección ``El decibelio'' del capítulo ``Fundamentos del Sonido''}
\end{itemize}


\section{Cuadros (mal llamados Tablas)}

Este es un ejemplo de inclusi\'on de cuadro en el texto. V\'ease el cuadro~\ref{tab:prueba}

\cuadro{|r||c|c|}{Cuadro de prueba}{tab:prueba}{elemento & elemento & elemento \\ elemento & elemento & elemento \\}