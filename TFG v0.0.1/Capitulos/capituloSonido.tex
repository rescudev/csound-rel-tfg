% !TEX root = ../proyect.tex

\chapter{Fundamentos del Sonido}

\section{Introducción}\label{sec:intro}
Este capítulo tiene como función dar una breve introducción a la teoría física del sonido, en concreto a los conceptos fundamentalmente necesarios para entender los ejemplos expuestos en esta guía de Csound.
Se presenta por ello como capítulo anexado o capítulo extra de modo que sirva de referencia rápida en otras partes del documento y de manera que un lector con manejo en estos términos pueda saltar su contenido cómodamente.


\section{¿Qué es el Audio Digital?}\label{sec:DigAud} 
Para definir el audio digital debemos empezar por saber qué es el audio o sonido:
''Sensación producida en el órgano del oído por el movimiento vibratorio de los cuerpos, transmitido por un medio elástico, como el aire.´´

\section{Plantillas}\label{sec:plantillas} 

Se encuentra disponible una plantilla LaTex en la secci\'on de documentos de la plataforma
de la aplicaci\'on de TfG de la ETSII 
\url{https://tfc.eii.us.es/TfG/}. Esa plantilla ha sido utilizada para preparar este documento.
Se espera en breve disponer de plantillas de ejemplo para las aplicaciones OpenOffice Writer y Office Word.

Nota: Las plantillas se proporcionan como ejemplo, las condiciones obligatorias son las que constan en este procedimiento.

\chapter{Ejemplos}\label{qwqwq} 
					
\section{Manejo de la bibliograf{\'\i}a}
En esta secci\'on mostramos brevemente ejemplos de referencia a la bibliograf{\'\i}a 
citando un libro~\cite{desousa}, un art{\'\i}culo~\cite{guiatitlesec} 
y una p\'agina web~\cite{informatica}. 

Se recuerda que son campos obligatorios
en todos los {\'\i}tems de la bibliograf{\'\i}a: 
autor(es), t{\'\i}tulo del libro o art{\'\i}culo 
% editorial (en su caso, tambi\'en revista) 
y a{\~n}o de publicaci\'on.
En el caso de p\'aginas web, 
es obligatoria la fecha de la \'ultima consulta. 
En general, la bibliograf{\'\i}a debe ayudar al lector a encontrar f\'acilmente los {\'\i}tems citados.

\section{C\'odigo}

En general se debe evitar incluir c\'odigo o seudoc\'odigo en la memoria. Si fuese preciso, se destacar\'a de forma
que sea f\'acilmente identificable y se indexar\'an los trozos de c\'odigo incluidos. Un ejemplo puede verse
a continuaci\'on.

\codigofuente{TeX}{Código de ejemplo en LaTeX}{codigo/macrotabla}

\section{Im\'agenes}

Este es un ejemplo de inclusi\'on de figura en el texto (v\'ease la figura~\ref{fig:ada}). 

\figura{0.5}{img/AdaLovelace}{Ada Lovelace}{fig:ada}{}

Figuras y cuadros se colocar\'an preferentemente tras el p\'arrafo en el que
son llamados por primera vez. Si no cupieran, se colocar\'an (en orden de preferencia):
\begin{itemize}
 \item Al final de la p\'agina en que se llaman
 \item Al principio de la siguiente p\'agina
 \item Al final del cap{\'\i}tulo
\end{itemize}
siempre respetando el orden de aparici\'on en el texto.

\section{Cuadros (mal llamados Tablas)}

Este es un ejemplo de inclusi\'on de cuadro en el texto. V\'ease el cuadro~\ref{tab:prueba}

\cuadro{|r||c|c|}{Cuadro de prueba}{tab:prueba}{elemento & elemento & elemento \\ elemento & elemento & elemento \\}