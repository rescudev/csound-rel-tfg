% !TEX root = ../proyect.tex

\chapter{Bibliografía Comentada}

Se presenta la bibliografía comentada de cada referencia usada en el presente documento.

\section{Csound FLOSS Manual}

 \begin{itemize}
 \item \textbf{Tipo de fuente}: Libro online
 
 \item \textbf{Última actualización}: Marzo de 2015
 
 \item \textbf{Dificultad}: Recomendado para principiantes.
 
 \item \textbf{Autor/es}: Joachim Heintz, Iain McCurdy, Andres Cabrera, Alex Hofmann, Alexandre Abrioux, Rory Walsh, Martin Neukom, Jim Aikin, Jacques Laplat, Menno Knevel, Bjorn Houdorf, Christopher Saunders, Oeyvind Brandtsegg, Oscar Pablo di Liscia, Peiman Khosravi, Steven Yi, Stefano Bonetti, Victor Lazzarini, Ed Costello, François Pinot, Tarmo Johannes, Nicholas Arner, Nikhil Singh, Richard Boulanger, Michael Gogins, Anton Kholomiov.
 \end{itemize}

\subsection{Referencia}

Comunidad de desarrolladores del lenguaje Csound (marzo de 2015). Csound FLOSS Manual. FLOSS Manuals. \url{http://write.flossmanuals.net/csound/preface/}

\subsection{Comentario}

El \textbf{Csound FLOSS Manual} es la principal fuente bibliográfica de este documento\footnote{Se ha sido especialmente extenso en el comentario de esta referencia por ser la principal fuente bibliográfica.}. Ha sido escrito por el núcleo de la comunidad de desarrolladores del código abierto de Csound y podría decirse sin miedo a equivocarse que es la fuente más completa y accesible para aprender sobre el lenguaje en la red.

Su estructura es la de un libro electrónico y la gran mayoría de sus capítulos cuenta con ejemplos de código completos e ilustrativos. Tiene además un capítulo dedicado por completo a explicar conceptos básicos sobre el sonido y el mundo de la edición sonora, aportando una buena base sólida del conocimiento previo que se recomendaría tener antes de probar un lenguaje dedicado al sonido.

Con todo y por estar realizado de mano directa por algunos de los principales autores del código fuente de Csound, \textbf{Csound FLOSS Manual} resulta ser un compendio actual de todos los conocimientos del lenguaje y es por ello que se recomienda encarecidamente su estudio y comprensión si se quiere aprender realmente los fundamentos de Csound.

\textbf{Csound FLOSS Manual} ha servido como base al presente documento tanto en su estructura como en su metodología para presentar los conocimiento.

En ocasiones \textbf{Csound FLOSS Manual} puede resultar algo tedioso precisamente por la completitud de sus ejemplos y explicaciones, cosa que se ha tratado de solventar en el presente documento al simplificar alguna explicaciones, pero es precisamente por ello que lo debido es recurrir al \textbf{Csound FLOSS Manual} para empezar a profundizar realmente en lo aprendido tras revisar los conocimientos aquí mostrados.

\subsection{Estructura}

\textbf{Csound FLOSS Manual} tiene 13 capítulos principales más 2 capítulos extra que resumiremos a continuación para tener una referencia útil acerca de dónde buscar para profundizar en cada concepto:

\begin{itemize}
 \item \textbf{01 BASICS}: Conceptos fundamentales sobre el sonido y su procesamiento. Muy útil incluso para el que sólo esté interesado en el mundo del sonido y no necesariamente en Csound.
 \item \textbf{02 QUICK START}: Información más básica sobre el lenguaje y sus IDEs. Cómo ejecutar programas, cómo exportarlos,  etc...
 \item \textbf{03 CSOUND LANGUAGE}: Fundamentos del lenguaje. Su sintaxis y las diferentes propiedades básicas como tipos de variables o funciones.
 \item \textbf{04 SOUND SYNTHESIS}: Conceptos físicos aplicados a la síntesis del sonido. Posee ejemplos más complejos para complementar las explicaciones algo más academicas.
 \item \textbf{05 SOUND MODIFICATION}: Capítulo dedicado principalmente a las capas de envoltura y efectos de sonido mediante filtros.
 \item \textbf{06 SAMPLES}: Capítulo dedicado a la lectura y escritura de archivos y a su consecuente aplicación en lo referente a los datos del sonido.
 \item \textbf{07 MIDI}: Dedicado a lo referente al MIDI (Musical Instrument Digital Interface) dando una extensa explicación acerca de cómo vincular nuestros instrumentos físicos o virtuales a nuestro código.
 \item \textbf{08 OTHER COMMUNICATION}: Capítulo corto pero interesante acerca de cómo combinar Csound con OSC y projectos con Arduino.
 \item \textbf{09 CSOUND IN OTHER APPLICATIONS}: Capítulo dedicado a explicar cómo combinar Csound con otros lenguajes y tecnologías dedicadas como PureData o Ableton Live.
 \item \textbf{10 CSOUND FRONTENDS}: Nos da una revisión media sobre los principales entornos de programación entre los que podemos elegir para usar Csound.
 \item \textbf{12 CSOUND AND OTHER PROGRAMMING LANGUAGES}: Como el propio título indica, se nos explica cómo y con qué sintaxis podemos compilar código Csound en diferentes lenguajes como Python  o Haskell.
 \item \textbf{13 EXTENDING CSOUND}: Capítulo corto que nos da una pequeña introducción acerca de cómo aportar al código abierto del lenguaje mediante, por ejemplo, la creación de nuevos opcodes.
 \item \textbf{OPCODE GUIDE}: Capítulo extra que aporta información más extensiva acerca del uso y funcionamiento de los opcodes.
 \item \textbf{APPENDIX}: Por último el apéndice, que aporta recomendaciones de nomenclatura, un glosario corto y una librería de enlaces con webs de información intereseante sobre Csound y el mundo del sonido.
\end{itemize}

\section{The Canonical Csound Reference Manual}

 \begin{itemize}
 \item \textbf{Tipo de fuente}: Manual online
 
 \item \textbf{Última actualización}: Enero de 2020
 
 \item \textbf{Dificultad}: Necesario conocimiento previo.
 
 \item \textbf{Autor/es}: Barry Vercoe, Comunidad de Csound.
 \end{itemize}

\subsection{Referencia}

Comunidad de Csound (Enero de 2020). The Canonical Csound Reference Manual. Csound. \url{https://csound.com/docs/manual/index.html}

\subsection{Comentario}

Manual más extenso

\subsection{Estructura}

 \begin{itemize}
 \item \textbf{1}:
 
 \item \textbf{a}:
 
 \item \textbf{b}:
 
 \item \textbf{c}: 
 \end{itemize}

\section{Cabbage Docs}

 \begin{itemize}
 \item \textbf{Tipo de fuente}: Documentación online
 
 \item \textbf{Última actualización}: Febrero de 2020
 
 \item \textbf{Dificultad}: Recomendado para principiantes.
 
 \item \textbf{Autor/es}: Rory Walsh, Iain McCurdy, Gordon Boyle.
 \end{itemize}

\subsection{Referencia}

Walsh R.(febrero de 2020). Cabbage Docs. Cabbage. \url{https://cabbageaudio.com/docs/introduction/}

\subsection{Comentario}

\textbf{Cabbage Docs} es la fuente bibliográfica de todo conociemiento referente al uso del IDE Cabbage y a sus Widgets. Ha sido escrita por Rory Walsh, principal desarrollador y autor de Cabbage, por lo que puede considerarse una fuente fiable de conocimiento.
Entre sus secciones ofrece una corta introducción a Csound que se recomienda usar como repaso al lenguaje, una sección dedicada a explicar el uso de Cabbage como IDE que cuenta con buenos ejemplos prácticos, y una sección a modo de biblioteca de referencia de uso de los diferentes Widgets existentes la cual se recomienda usar como API de ejemplos de uso específico del IDE.

\subsection{Estructura}

\textbf{Cabbage Docs} tiene cuatro secciones:
 \begin{itemize}
 \item \textbf{Beginners(Csound)}: Da una introducción al lenguaje Csound.
 
 \item \textbf{Using Cabbage}: Da una introducción a cómo sacar provecho del IDE Cabbage al usar Csound
 
 \item \textbf{Advanced Features}: Extiende algo más sobre el uso de Csound y sus conceptos de uso avanzado.
 
 \item \textbf{Cabbage Widgets}: Sirve de biblioteca de referencia de los diferentes Widgets aportados por Cabbage.
 \end{itemize} 