% !TEX root = ../proyect.tex

\chapter{Bibliografía Comentada}

Se presenta la bibliografía comentada de cada referencia usada en el presente documento.

\section{Csound FLOSS Manual}

 \begin{itemize}
 \item \textbf{Tipo de fuente}: Libro online
 
 \item \textbf{Última actualización}: Marzo de 2015
 
 \item \textbf{Dificultad}: Recomendado para principiantes.
 
 \item \textbf{Autor/es}: Joachim Heintz, Iain McCurdy, Andres Cabrera, Alex Hofmann, Alexandre Abrioux, Rory Walsh, Martin Neukom, Jim Aikin, Jacques Laplat, Menno Knevel, Bjorn Houdorf, Christopher Saunders, Oeyvind Brandtsegg, Oscar Pablo di Liscia, Peiman Khosravi, Steven Yi, Stefano Bonetti, Victor Lazzarini, Ed Costello, François Pinot, Tarmo Johannes, Nicholas Arner, Nikhil Singh, Richard Boulanger, Michael Gogins, Anton Kholomiov.
 \end{itemize}

\subsection{Referencia}

Comunidad de desarrolladores del lenguaje Csound (marzo de 2015). Csound FLOSS Manual. FLOSS Manuals. \url{http://write.flossmanuals.net/csound/preface/}

\subsection{Comentario}

El \textbf{Csound FLOSS Manual} es la principal fuente bibliográfica de este documento\footnote{Se ha sido especialmente extenso en el comentario de esta referencia por ser la principal fuente bibliográfica.}. Ha sido escrito por el núcleo de la comunidad de desarrolladores del código abierto de Csound y podría decirse sin miedo a equivocarse que es la fuente más completa y accesible para aprender sobre el lenguaje en la red.

Su estructura es la de un libro electrónico y la gran mayoría de sus capítulos cuenta con ejemplos de código completos e ilustrativos. Tiene además un capítulo dedicado por completo a explicar conceptos básicos sobre el sonido y el mundo de la edición sonora, aportando una buena base sólida del conocimiento previo que se recomendaría tener antes de probar un lenguaje dedicado al sonido.

Con todo y por estar realizado de mano directa por algunos de los principales autores del código fuente de Csound, \textbf{Csound FLOSS Manual} resulta ser un compendio actual de todos los conocimientos del lenguaje y es por ello que se recomienda encarecidamente su estudio y comprensión si se quiere aprender realmente los fundamentos de Csound.

\textbf{Csound FLOSS Manual} ha servido como base al presente documento tanto en su estructura como en su metodología para presentar los conocimiento.

En ocasiones \textbf{Csound FLOSS Manual} puede resultar algo tedioso precisamente por la completitud de sus ejemplos y explicaciones, cosa que se ha tratado de solventar en el presente documento al simplificar algunas explicaciones, pero es precisamente por ello que lo debido es recurrir al \textbf{Csound FLOSS Manual} para empezar a profundizar realmente en lo aprendido tras revisar los conocimientos aquí mostrados.

\subsection{Estructura}

\textbf{Csound FLOSS Manual} tiene 13 capítulos principales más 2 capítulos extra que resumiremos a continuación para tener una referencia útil acerca de dónde buscar para profundizar en cada concepto:

\begin{itemize}
 \item \textbf{01 BASICS}: Conceptos fundamentales sobre el sonido y su procesamiento. Muy útil incluso para el que sólo esté interesado en el mundo del sonido y no necesariamente en Csound.
 \item \textbf{02 QUICK START}: Información más básica sobre el lenguaje y sus IDEs. Cómo ejecutar programas, cómo exportarlos,  etc...
 \item \textbf{03 CSOUND LANGUAGE}: Fundamentos del lenguaje. Su sintaxis y las diferentes propiedades básicas como tipos de variables o funciones.
 \item \textbf{04 SOUND SYNTHESIS}: Conceptos físicos aplicados a la síntesis del sonido. Posee ejemplos más complejos para complementar las explicaciones algo más academicas.
 \item \textbf{05 SOUND MODIFICATION}: Capítulo dedicado principalmente a las capas de envoltura y efectos de sonido mediante filtros.
 \item \textbf{06 SAMPLES}: Capítulo dedicado a la lectura y escritura de archivos y a su consecuente aplicación en lo referente a los datos del sonido.
 \item \textbf{07 MIDI}: Dedicado a lo referente al MIDI (Musical Instrument Digital Interface) dando una extensa explicación acerca de cómo vincular nuestros instrumentos físicos o virtuales a nuestro código.
 \item \textbf{08 OTHER COMMUNICATION}: Capítulo corto pero interesante acerca de cómo combinar Csound con OSC y projectos con Arduino.
 \item \textbf{09 CSOUND IN OTHER APPLICATIONS}: Capítulo dedicado a explicar cómo combinar Csound con otros lenguajes y tecnologías dedicadas como PureData o Ableton Live.
 \item \textbf{10 CSOUND FRONTENDS}: Nos da una revisión media sobre los principales entornos de programación entre los que podemos elegir para usar Csound.
 \item \textbf{12 CSOUND AND OTHER PROGRAMMING LANGUAGES}: Como el propio título indica, se nos explica cómo y con qué sintaxis podemos compilar código Csound en diferentes lenguajes como Python  o Haskell.
 \item \textbf{13 EXTENDING CSOUND}: Capítulo corto que nos da una pequeña introducción acerca de cómo aportar al código abierto del lenguaje mediante, por ejemplo, la creación de nuevos opcodes.
 \item \textbf{OPCODE GUIDE}: Capítulo extra que aporta información más extensiva acerca del uso y funcionamiento de los opcodes.
 \item \textbf{APPENDIX}: Por último el apéndice, que aporta recomendaciones de nomenclatura, un glosario corto y una librería de enlaces con webs de información intereseante sobre Csound y el mundo del sonido.
\end{itemize}

\section{The Canonical Csound Reference Manual}

 \begin{itemize}
 \item \textbf{Tipo de fuente}: Manual online
 
 \item \textbf{Última actualización}: Enero de 2020
 
 \item \textbf{Dificultad}: Necesario conocimiento previo.
 
 \item \textbf{Autor/es}: Barry Vercoe, Comunidad de Csound.
 \end{itemize}

\subsection{Referencia}

Comunidad de Csound (Enero de 2020). The Canonical Csound Reference Manual. Csound. \url{https://csound.com/docs/manual/index.html}

\subsection{Comentario}

\textbf{The Canonical Csound Reference Manual} es el manual más extenso de información sobre Csound en la red. Es por ello algo menos accesible para principiantes en el lenguaje pero el mejor compendio de referencias si necesitamos información acerca de, por ejemplo, un opcode concreto. Este manual ha sido creado por los desarrolladores del lenguaje Csound y se actualiza conjuntamente con el propio lenguaje para reflejar las nuevas características del código fuente.

En referencia a este documento, ha servido para completar y complementar conocimientos en el uso y sintaxis concretos de algunos métodos y opcodes, y descripción de algunos conceptos del lenguaje que sólo pueden encontrarse en el propio manual.
Se recomienda su uso como guía de referencia siempre que se sepa previamente qué se está buscando.

Su contenido se divide en tres partes principales, una primera describiendo conceptos de Csound, una segunda extendiendo este conocimiento e introduciendo conceptos como la generación y edición de señales, y una tercera a modo de biblioteca de referencia de todos los opcodes (incluyendo los obsoletos) disponibles en Csound. Posee además una biblioteca de archivos descargables con cientos de ejemplos de programación y apartados con datos útiles como una tabla de conversión de valores de onda a nota musical.

\subsection{Estructura}

 \begin{itemize}
 \item \textbf{I. Overview}: Da una base fundamental del uso del lenguaje, como punto a favor posee un enlace a una página de la misma guía cada vez que se menciona algún opcode o palabra reservada por lo que queda bien estructurado.
 
 \item \textbf{II. Opcodes Overview}: Ofrece información extensa acerca del uso de los principales opcodes de Csound dividiendo el capítulo según sus tipos (De control de instrucciones, de edición de señales de audio, relacionadas con el MIDI, etc...)
 
 \item \textbf{III. Reference}: Biblioteca de referencia de los diferentes opcodes y operadores del lenguaje que se recomienda tener siempre a mano. El resto de manual hace constante referencia a esta sección.
 
 \item \textbf{Apéndices}: Ofrece información útil sobre diversos temas relacionados al sonido o a la sintaxis concreta de Csound.
 \end{itemize}

\section{Cabbage Docs}

 \begin{itemize}
 \item \textbf{Tipo de fuente}: Documentación online
 
 \item \textbf{Última actualización}: Febrero de 2020
 
 \item \textbf{Dificultad}: Recomendado para principiantes.
 
 \item \textbf{Autor/es}: Rory Walsh, Iain McCurdy, Gordon Boyle.
 \end{itemize}

\subsection{Referencia}

Walsh R.(febrero de 2020). Cabbage Docs. Cabbage. \url{https://cabbageaudio.com/docs/introduction/}

\subsection{Comentario}

\textbf{Cabbage Docs} es la fuente bibliográfica de todo conociemiento referente al uso del IDE Cabbage y a sus Widgets. Ha sido escrita por Rory Walsh, principal desarrollador y autor de Cabbage, por lo que puede considerarse una fuente fiable de conocimiento.
Entre sus secciones ofrece una corta introducción a Csound que se recomienda usar como repaso al lenguaje, una sección dedicada a explicar el uso de Cabbage como IDE que cuenta con buenos ejemplos prácticos, y una sección a modo de biblioteca de referencia de uso de los diferentes Widgets existentes la cual se recomienda usar como API de ejemplos de uso específico del IDE.

\subsection{Estructura}

\textbf{Cabbage Docs} tiene cuatro secciones:
 \begin{itemize}
 \item \textbf{Beginners(Csound)}: Da una introducción al lenguaje Csound.
 
 \item \textbf{Using Cabbage}: Da una introducción a cómo sacar provecho del IDE Cabbage al usar Csound
 
 \item \textbf{Advanced Features}: Extiende algo más sobre el uso de Csound y sus conceptos de uso avanzado.
 
 \item \textbf{Cabbage Widgets}: Sirve de biblioteca de referencia de los diferentes Widgets aportados por Cabbage.
 \end{itemize} 
 
\section{CS Csound: Página oficial}

 \begin{itemize}
 \item \textbf{Tipo de fuente}: Página Web
 
 \item \textbf{Última actualización}: Junio de 2020
 
 \item \textbf{Dificultad}: Recomendado para principiantes.
 
 \item \textbf{Autor/es}: Comunidad de Csound.
 \end{itemize}

\subsection{Referencia}

Comunidad de Csound(Junio de 2020). CS Csound. Csound. \url{https://csound.com/index.html}

\subsection{Comentario}

Página web oficial de Csound en la que se comparten noticias y nuevas publicaciones. Tiene una muy buena introducción al lenguaje la cual ha servido de base para el primer capítulo de este documento. Se recomienda el uso de esta web como nexo de los diferentes contenidos de Csound puesto que también pueden encontrarse enlaces dedicados a la referencia de otros portales de contenido.

\section{Repositorio csound-live-code}

 \begin{itemize}
 \item \textbf{Tipo de fuente}: Documentación de repositorio
 
 \item \textbf{Última actualización}: Mayo de 2019
 
 \item \textbf{Dificultad}: Necesario conocimiento previo.
 
 \item \textbf{Autor/es}: Steven Yi.
 \end{itemize}

\subsection{Referencia}

Yi S.(Mayo de 2019). csound-live-code doc. Github. \url{https://github.com/kunstmusik/csound-live-code/blob/master/doc/intro.md}

\subsection{Comentario}

Documentación del repositorio \textbf{csound-live-code} del desarrollador Steven Yi, colaborador de Csound y autor de blue y la mayor parte del contenido sobre \textbf{Live Coding} en Csound.

La documentación del repositorio ha servido como base para el capítulo ``Haciendo música en directo'' de este documento.

Se recomienda revisar esta fuente para ahondar en el concepto de codificación en vivo de Csound aunque sería preferible llegar a este contenido con conocimientos previos sobre el lenguaje.
 
\section{Canal de youtube: Steven Yi}

 \begin{itemize}
 \item \textbf{Tipo de fuente}: Canal de Youtube
 
 \item \textbf{Última actualización}: Junio de 2020
 
 \item \textbf{Dificultad}: Necesario bastante conocimiento previo.
 
 \item \textbf{Autor/es}: Steven Yi.
 \end{itemize}

\subsection{Referencia}

Yi S.(Junio de 2020). Canal de youtube: Steven Yi. Youtube. \url{https://www.youtube.com/channel/UCiO3nb3sN5GsZUIyxrgXh0Q}

\subsection{Comentario}

Canal personal de youtube de Seteven Yi al que sube ejemplos de uso de \textbf{Live Coding Csound} haciendo música en directo. Se recomienda bastante conocimiento previo antes de visitar el canal puesto que los ejemplos son únicamente visuales y no van dirigidos como contenido pedagógico sino más bien divulgativo acerca de lo que se puede hacer con Csound.

Respecto a este documento ha servido como base de algunos de los ejemplos de código mostrados.

\section{Canal de youtube: Rory Walsh}

 \begin{itemize}
 \item \textbf{Tipo de fuente}: Canal de Youtube
 
 \item \textbf{Última actualización}: Mayo de 2019
 
 \item \textbf{Dificultad}: Recomendado para principiantes.
 
 \item \textbf{Autor/es}: Rory Walsh.
 \end{itemize}

\subsection{Referencia}

Walsh R.(Mayo de 2019). Canal de youtube: Rory Walsh. Youtube. \url{https://www.youtube.com/channel/UCWB8axin-qfcWlMt-8LhrHg}

\subsection{Comentario}

Canal personal de youtube de Rory Walsh, Creador de Cabbage. 

Respecto a este documento ha servido como base de algunos de los ejemplos de código mostrados. En este canal podemos encontrar una introducción en vídeo a la programación de Csound y al uso de Cabbage, es por lo tanto altamente recomendable para principiantes como método algo más dinámico que el texto con ejemplos.

Respecto a este documento ha servido para fundamentar el capítulo ``Cabbage: Guía de uso'' y para detallar algunas partes de los ejemplos mostrados.

\section{Repositorio csoundAPI examples}

 \begin{itemize}
 \item \textbf{Tipo de fuente}: Documentación de repositorio
 
 \item \textbf{Última actualización}: Agosto de 2019
 
 \item \textbf{Dificultad}: Necesario Bastante conocimiento previo.
 
 \item \textbf{Autor/es}: Comunidad de desarrolladores de Csound.
 \end{itemize}

\subsection{Referencia}

Comunidad de desarrolladores de Csound(Agosto de 2019). csoundAPI\_ examples. Github. \url{https://github.com/csound/csoundAPI\_ examples}

\subsection{Comentario}

Repositorio con ejemplos del uso de Csound en otros lenguajes. La documentación para ejecutar el código y hacerlo funcionar es algo escasa pero los ejemplos son unitarios y completos, contando con comentarios exhaustivos acompañando cada línea.

Este repositorio ha sido la fuente principal de información para el capítulo ``Csound y otros lenguajes''. Se recomienda su uso una vez se haya dominado una base sólida del lenguaje y quiera extenderse el conocimiento acerca de qué puede ofrecer Csound.
\pagebreak

\section{Otras referencias}

 \begin{itemize}
 \item Roth D.(Sin fecha). GAIN VS VOLUME: WHAT’S THE DIFFERENCE?. musicianonamission. \url{https://www.musicianonamission.com/gain-vs-volume/}
 
 \item Steven J Greenfield(Enero 2015). Is it possible to generate a sine wave of different positive and negative amplitude?. quora. \url{https://www.quora.com/Is-it-possible-to-generate-a-sine-wave-of-different-positive-and-negative-amplitude}
 
 \item The Editors of Encyclopaedia Britannica.(Sin fecha). Envelope. britannica. \url{https://www.britannica.com/science/envelope-sound}
  
 \item Sin referencia a autor.(Sin fecha). El Concepto de Fase. azimadli. \url{http://azimadli.com/vibman-spanish/elconceptodefase.htm}
 
 \item Comunidad wikibooks.(Septiembre 2018). Sound in the Digital Domain. wikibooks. \url{https://en.wikibooks.org/wiki/Sound_in_the_Digital_Domain}
  
 \item Paredes F.(Sin fecha). Aliasing. musicianonamission. \url{https://musicaysonido.es/2019/02/11/aliasing/}
   
 \item Sin referencia a autor.(Sin fecha). Audio Digital. cenart. \url{http://cmm.cenart.gob.mx/tallerdeaudio/cursos/cursoardour/Teoria_y_tecnicas/Audiodigital.html}
    
 \item Byrne M.(Julio 2015). Know Your Language: Csound May Be Ancient, But It's the Audio Hacking Future. vice. \url{https://www.vice.com/en_us/article/pga53v/know-your-language-csound-may-be-ancient-but-its-the-audio-hacking-future}
     
 \item Porter D.(Julio 2015). Driving force behind laptop for each child. nzherald. \url{https://www.nzherald.co.nz/bay-of-plenty-times/business/news/article.cfm?c_id=1503347&objectid=11482590}
 \end{itemize}
