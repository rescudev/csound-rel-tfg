% !TEX root = ../proyect.tex

\chapter{Introducción al Lenguaje y Fundamentos del Sonido}\label{cap1}
\section{¿Qué es Csound?}\label{sec:intro}

Csound es un lenguaje de programación de alto nivel orientado a objetos dedicado a la síntesis, edición y producción de sonido. Su sintaxis es concreta y su compilador está codificado en lenguaje C. Entre los usos prácticos del lenguaje podemos encontrar ejemplos en la página oficial (https://csound.com/projects.html) de proyectos dedicados a la síntesis de música en directo, edición de sonido mediante efectos generados programáticamente y creación de interfaces VST o instrumentos virtuales entre otros.

Podemos referirnos a Csound como un ``Compilador de Sonido''.

\section{¿Por qué usar Csound?}\label{sec:intro}

Hay muchas buenas razones para usar Csound. Csound es software libre de código abierto con licencia LGPL, está en constante desarrollo y tiene una comunidad de desarrolladores que crece día a día,proporciona además un punto de conexión entre la disciplina informática y el ámbito de la música y el sonido. 
Es además una gran herramienta científica al facilitar la exposición y experimentación de conceptos relacionados a las ondas del sonido, con una amplia librería de funciones y objetos útiles para ello. 
En lo que se refiere a la composición musical, Csound se ha usado principalmente para generar música electrónica a lo largo de su historia aunque podemos encontrar a compositores de cualquier género músical o tipo de instrumento sacándole provecho al lenguaje. No es de extrañar vistas las tendencias actuales en el mundo de la música, donde para triunfar a nivel multitudinario es prácticamente imprescindible contar con un buen productor que sepa embellecer el sonido.
Es frecuente ver a usuarios del lenguaje interpretando su música en directo con ayuda directa de éste.
Y por último debo destacar que Csound es compatible con los principales sistemas operativos del mercado, desde Windows a iOS pasándo por distrbuciones Linux. Y como veremos más adelante en esta guía, sus funciones pueden llamarse desde el código de otros lenguajes como Python, java o C.
\pagebreak

\section{Breve historia de Csound}\label{sec:intro}

De manera resumida, Csound fue desarrollado en un principio por Barry Vercoe (véase la figura 1.1) en 1985 en el MIT\footnote{Massachusetts Institute of Technology} Media Lab. Y desde la década de los años 1990, una amplia variedad de desarrolladores ha colaborado a su código abierto, aportando además documentación, ejemplos y artículos sobre el lenguaje.

\figura{0.5}{img/1.1-Barry-Vercoe}{Barry Vercoe}{fig:barry}{}

Para hablar de los verdaderos orígenes de Csound debemos remontarnos a la década de los años 1970, a los orígenes de la producción informática de sonido. 
Matt Mathews creó MUSIC, el primer lenguaje informático para la generación de ondas de audio digital. En éste se basarían sus posteriores iteraciones: Music1, Music2, Music3, Music4, Music4B... Hasta llegar a Music11, desarrollado por el mentado Barry Vercoe, y del cual Csound es sucesor directo.
Posteriormente el desarrollo del lenguaje ha continuado gracias a su comunidad con John Fitch de la University of Bath a la cabeza y como dueño del repositorio de código abierto en la plataforma GitHub.
En la actualidad se realizan periódicamente las ICSC\footnote{International Csound Conference}, conferencias dedicadas a Csound a nivel internacional de manera periódica, siendo la última fecha de realización el 27 de septiembre de 2019 en la actualidad del presente documento.

\section{Las características del lenguaje}\label{sec:intro}

A continuación se muestra una lista de las características principales y técnicas del lenguaje:
\begin{itemize}
 \item Usado para la síntesis, edición y análisis de sonido y música.
 \item Lenguaje de código abierto.
 \item Programación funcional.
 \item Licencia de distribución LGPL.
 \item Orientado a objetos.
 \item Compilador programado en lenguaje C.
 \item 30 años de desarrollo.
 \item Compatibilidad con otros lenguajes y retrocompatibilidad de versiones.
\end{itemize}



\section{Alternativas a Csound}\label{sec:intro}

La Normativa académica de los Trabajos fin de Grado indica:

\emph{Como norma general, el TfG deberá estar escrito y ser expuesto oralmente en castellano.
Podrá también estar escrito y ser expuesto en inglés, previa solicitud.}

En cualquier caso, la memoria debe respetar los usos y costumbres del idioma en que sea escrita.
Debe prestarse especial atenci\'on al guionado de las palabras, debido a que muchas aplicaciones inform\'aticas
usan el propio del ingl\'es y no el del castellano.

En cuanto a la portada, se debe utilizar la portada oficial de la ETSII. Un ejemplo de la misma puede encontrarse al 
principio (como portada) de este documento.


 
