% !TEX root = ../proyect.tex

\chapter{Introducción al Lenguaje}\label{cap1}
\section{¿Qué es Csound?}\label{sec:intro}

Csound es un lenguaje de programación de alto nivel orientado a objetos dedicado a la síntesis, edición y producción de sonido. Su sintaxis es concreta y su compilador está codificado en lenguaje C. Entre los usos prácticos del lenguaje podemos encontrar ejemplos en la página oficial (https://csound.com/projects.html) de proyectos dedicados a la síntesis de música en directo, edición de sonido mediante efectos generados programáticamente y creación de interfaces VST o instrumentos virtuales entre otros.
Podemos referirnos a Csound como un "Compilador de Sonido".


\section{¿Por qué usar Csound?}\label{sec:intro}

Hay muchas buenas razones para usar Csound. Csound es software libre de código abierto con licencia LGPL, está en constante desarrollo y tiene una comunidad de desarrolladores que crece día a día,proporciona además un punto de conexión entre la disciplina informática y el ámbito de la música y el sonido. 
Es además una gran herramienta científica al facilitar la exposición y experimentación de conceptos relacionados a las ondas del sonido, con una amplia librería de funciones y objetos útiles para ello. 


\section{Cuerpo y tipos de letra}\label{sec:letra}

Se recomienda utilizar, dada su simplicidad, claridad y legibilidad, los tipos de letra
Arial (preferiblemente en caja alta) o Helv\'etica, en un tamaño para el cuerpo del texto de 11 pt,
con un interlineado sencillo o de 1,5.

Los t{\'\i}tulos de cap{\'\i}tulos, secciones y subsecciones, as{\'\i} como las notas al pie de texto\footnote{Por favor, no abusad de las notas a pie de texto}
y las cabeceras o pies de cuadros, figuras y trozos de c\'odigo quedan a libertad del redactor de la memoria.
Sin embargo, es buena idea que los t{\'\i}tulos tengan un tamaño igual o superior a 11pt. y los pies y cabeceras
sean de tama{\~n}o igual o inferior a 11pt. Se ruega encarecidamente que, en lo posible, se evite el \underline{subrayado}.
Este puede sustituirse por el uso de \textbf{negritas} o \textit{cursivas} o por el cambio de formato del tipo de letra
(Se recomienda prestar atenci\'on y no abusar del cambio de {\color{blue}{color}}, 
pues puede dar problemas de accesibilidad).

 

\section{Márgenes y párrafos}\label{sec:margenes}

Es necesario configurar la página seleccionando los márgenes siguientes:
\begin{itemize}
 \item Márgenes superior e inferior: 2,5 cm. 
 \item Márgenes laterales (izquierdo y derecho): 3 cm.
\end{itemize}

En el caso de no usar cabeceras, las páginas deber\'an ir numeradas en el centro del pie.
Si se usan cabeceras, no se utilizarán estas en las p\'aginas que comienzan capítulo, las cuales
se numerar\'an en el centro del pie. Las restantes p\'aginas pueden ir numeradas en el pie o en la cabecera,
pero deber\'an mantener coherencia de formato a lo largo de todo el cap{\'\i}tulo.

Las p\'aginas previas al cuerpo de la memoria del Trabajo fin de Grado (agradecimientos, resumen, {\'\i}ndices,\dots) 
pueden no numerarse o
numerarse independientemente de la misma, en cuyo caso se numerar\'an con n\'umeros romanos. 
Se recuerda que los n\'umeros romanos
se escriben con letras may\'usculas (la numeraci\'on i,ii,iii,iv\dots es propia del idioma ingl\'es 
y no es admisible en espa{\~n}ol, ni siquiera para enumeraciones)

Los p\'arrafos comenzar\'an con sangrado. El espacio entre los mismos no debe ser excesivo.

\section{Lengua}\label{sec:lengua}

La Normativa académica de los Trabajos fin de Grado indica:

\emph{Como norma general, el TfG deberá estar escrito y ser expuesto oralmente en castellano.
Podrá también estar escrito y ser expuesto en inglés, previa solicitud.}

En cualquier caso, la memoria debe respetar los usos y costumbres del idioma en que sea escrita.
Debe prestarse especial atenci\'on al guionado de las palabras, debido a que muchas aplicaciones inform\'aticas
usan el propio del ingl\'es y no el del castellano.

En cuanto a la portada, se debe utilizar la portada oficial de la ETSII. Un ejemplo de la misma puede encontrarse al 
principio (como portada) de este documento.


 
