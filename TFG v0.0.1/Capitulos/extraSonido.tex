% !TEX root = ../proyect.tex

\chapter{Fundamentos del Sonido}

\section{Introducción}\label{sec:intro}
Este capítulo tiene como función dar una breve introducción a la teoría física del sonido, en concreto a los conceptos fundamentalmente necesarios para entender los ejemplos expuestos en esta guía de Csound.
Se presenta por ello como capítulo anexado o capítulo extra de modo que sirva de referencia rápida en otras partes del documento y de manera que un lector con manejo en estos términos pueda saltar su contenido cómodamente.


\section{El Audio Digital}\label{sec:DigAud} 
Para definir el audio digital debemos empezar por saber qué es el sonido:\bigskip

\subsection{¿Qué es el sonido y cómo se transmite?}

\textbf{Sonido}: ``\textsl{Sensación producida en el órgano del oído por el movimiento vibratorio de los cuerpos, transmitido por un medio elástico, como el aire.}''\footnote{REAL ACADEMIA ESPAÑOLA: Diccionario de la lengua española, 23.ª ed., [versión 23.3 en línea]. <https://dle.rae.es> [05 de julio de 2020].}\bigskip

A ese movimiento vibratorio que se transmite y viaja por el medio podemos llamarlo ``Onda de Sonido''. Y la forma más simple de describir un movimiento vibratorio, es decir, la onda más simple de todas; es mediante la forma senoidal:

\figura{0.8}{img/S.1-senoidal}{Onda Senoidal}{fig:Senoidal}{}

Como sabemos, los medios transmisores están formador por moléculas que ocupan un determinado espacio. Podemos por lo tanto imaginar a una molécula que describe el movimiento vibratorio descrito anteriormente. Podríamos también dedir que cuando la molécula sobrepasa el punto inicial o punto 0 que definimos en la gráfica, la molécula está empujando al resto de moléculas que encuentra en su camino. De forma análoga, cuando la posición de la molécula tiene un valor menor al inicial decimos que la molécula está tirando del resto de moléculas de su entorno.

De esta manera se produce la transmisión del sonido.

\subsection{La onda de sonido y sus características}\label{sec:Ondas} 

Quedaba definida la onda de sonido en el apartado anterior. Si a continuación le añadimos la información de esa vibración de la que hablábamos es constante se producirá lo que llamamos ``Onda Periódica''.

Toda onda periódica posee 4 características:

\begin{itemize}
	\item \textbf{Periodo}: Es la cantidad de tiempo que tarda la forma de la onda en repetirse, lo llamaremos T y lo expresaremos en segundos.
	 \item \textbf{Amplitud}: Distancia máxima de los puntos de la onda respecto a la posición de eje Y (eje ``Tiempo'' en la figura). Podemos definirla también como la fuerza con la que las moléculas del medio consiguen empujar o tirar del resto de moléculas de su entorno.
\figura{0.6}{img/S.2-Periodo}{Periodo y Amplitud}{fig:Periodo}{}
    \item \textbf{Frecuencia}: La frecuencia de una onda expresa la cantidad numérica de veces que repite su movimiento durante un tiempo determinado. Si la definimos respecto al periodo decimos que es la cantidad numérica de periodos (o repeticiones de la forma de la onda) que ocurren durante un segundo. Se mide en Hercios o Hz, la representaremos con f y podremos calcularla fácilmento puesto que es la inversa del periodo:
    \begin{itemize}
    \item Frecuencia = 1/Periodo
    \item Periodo = 1/Frecuencia   
    \end{itemize}
    \item \textbf{Longitud de Onda}: Se trata de la distancia que va del punto inicial al punto final del recorrido de la onda marcada por un periodo. Se mide en metros.
    \item \textbf{Fase}: Punto de partida de la onda. Podemos observar al representar la onda en un gráfico y fijarnos en que el valor inicial del eje Y que no tiene que ser nesesariamente 0.
\end{itemize} 

\subsection{El Sampleo y Sample Rate}\label{sec:sr} 

Para representar de manera digital una onda de sonido acústica necesitamos convertir sus valores analógicos a digitales. Esta onda tendrá un valor distinto por cada instante de tiempo y para conseguir recoger de alguna manera estos datos en un computador necesitaremos el concepto de sampleo.

El sampleo consiste en recoger un número determinado de valores en formato digital de una onda sonora por cada segundo de duración.

\figura{0.8}{img/S.3-Sampleo}{Ejemplo de sampleo de una onda}{fig:Sampleo}{}

Aquí entra el concepto de ``sample rate'', que determina la frecuencia de muestreo en la recogida de datos de la onda. En la figura anterior podemos observaruna misma onda de sonido a la que se le realiza un sampleo primero con sample rate menor (izquierda), y con un sample rate mayor (derecha).

Por último debe destacarse que un sample rate mayor no implica necesariamente una coversión digital más feaciente para nosotros a un nivel pragmático puesto que el nivel máximo de frecuencia de datos sonoros que puede captar en oído humano ronda los 20Khz. 

Contamos además con que debe respetarse el \textbf{Teorema de muestreo de Nyquist-Shannon}:\bigskip

\textsl{``Para representar una onda de manera digital que contenga frecuencias de hasta X Hz, es necesario usar un sample rate de al menos 2X muestras por segundo.''}\bigskip

De otra manera, los valores digitales no representarían la onda de manera correcta dando lugar al aliasing y al muestreo incorrecto.

Se muestra a continuación una figura de ejemplo de sampleo en la que se toma un sample rate de 40000Hz. De la primera onda, que es de 10KHz observamos que recogemos los datos suficiente cada segundo como para captar toda la información contenida en ella. De hecho podemos observar también que con un sample rate de 20KHz en lugar de 40KHz también captaríamos toda la información necesaria como para recoger la onda al completo en su formato digital cumpliéndose así el Teorema de muestreo de Nyquist-Shannon.

Sin embargo la onda del segundo gráfico tiene una frecuencia de 30KHz y esto implica que con un sample rate de 40KHz como el mostrado, daría resultado una conversión de onda errónea. Dando incluso para este ejemplo una muestra de onda digital idéntica a la del primer gráfico. Sería necesario un sample rate de al menos 60KHz para recoger feacientemente los datos y conseguir una conversión digital satisfactoria de la onda.\pagebreak

\figura{0.8}{img/S.4-Aliasing}{Un mismo sample rate para ondas distintas}{fig:difSamples}{}

\section{Conceptos interesantes}\label{sec:Conceptos} 

Hablemos a continuación de algunos conceptos no necesariamente intrínsecos al sonido digital pero interesantes para entender mejor los conceptos necesarios para el uso adecuado de Csound.

\subsection{El decibelio}\label{sec:db} 

El decibelio o `db'\ es una unidad de medida que representa la intensidad de un sonido. Es la décima parte de un belio y siempre que hablamos de decibelio lo hacemos respecto a un valor de referencia preestablecido de intensidad, normalmente el marcado por el límite por debajo del oído humano en su capacidad para oír: \( I_{0} = 10^{-12}W/m^2 \) que se da en los 1000Hz.

La fórmula para calcular los decibelios es: \( 10*log_{10}* \frac{I}{I_{0}} \), es por tanto una fórmula logarítmica que depende de su valor de referencia 
\( I_{0} \). Para una relación \( \frac{I}{I_{0}} \) de 10 tenemos 10db, para una relación de 100 tendremos 20db, para 1000 30db, etc...

Un dato útil a tener en cuenta es que al doblar la amplitud de una onda de sonido obtenemos una diferencia de +6db. De manera análoga si partimos por la mitad el valor de la amplitud de la onda obtenemos un cambio de -6db.

\subsection{El ADSR}\label{sec:ADSR} 

\textbf{ADSR} son las siglas de \textbf{A}ttack, \textbf{D}ecay, \textbf{S}ustain y \textbf{R}elease en una onda y conforman la opción más común de envelope o envolvente sonoro, es decir, proporcionan parámetros para poder controlar una onda de sonido.

Veamos cada uno de estos parámetros para entenderlos en conjunto:

\begin{itemize}
    \item \textbf{Attack}: El Attack o Ataque sería lo ocurrido antes de que la onda decaiga y se estabilice. Por ejemplo, un golpe de platillo produce un sonido con mucho ataque, el sonido de una nota tocada en una flauta dulce tendría normalmente poco ataque.
    \item \textbf{Decay}: El decay o decaimiento es lo sucedido entre el ataque ensu máximo punto y la fase estable de la onda. Al rasgar las cuerdas de una guitarra con una púa se produce un sonido con bastante decay.
    \item \textbf{Sustain}: El sustain o sostenibilidad es lo referido a la parte estable de la onda, su duración e intensidad máxima. Un golpe de caja tiene poco sustain, una nota tocada al aire en la cuerda de un bajo eléctrico tiene mucho sustain.
    \item \textbf{Release}: El release es la parte de la onda comprendida entre la fase estable o de sustain y la llegada al valor 0 de intensidad. Los instrumentos de cuerda tienen por lo gerenar una fase de release notable.
\end{itemize}

\figura{0.8}{img/S.5-ADSR}{El envolvente ADSR de una onda}{fig:difSamples}{}

Es común en el sonido digital poder modificar estos parámetros a placer en instrumentos como los sintetizadores o en prácticamente cualquier instrumento con componentes eléctricos.

\subsection{El Cutoff y la resonancia}\label{sec:Cutoff} 

Veamos este par de conceptos útiles para completar y mejorar nuestro sonido:

\begin{itemize}
    \item \textbf{Cuttoff}: Se trata de un filtro de frecuencias. Normalmente se usa para bloquear determinados rangos de frecuencias altas, siendo esto de tipo LP (LOW PASS) aunque en instrumentos modernos pueden encontrar Cutoffs de tipo HP (HIGH PASS) que bloquearían en este caso frecuencias más bajas de lo deseado.
    \item \textbf{Resonancia}: Es el efecto que ocurre por ejemplo cuando la mesa que sujeta unos altavoces en marcha empieza a vibrar mientras algunos sonidos son producidos, querrá decir que el sonido que sale de esos altavoces producen una frecuencia coincidente con la frecuencia de resonancia de la tabla de la mesa y el sonido por tanto se ve amplificado. En el audio digital puede encontrarse la resonancia como un efecto más a añadir a nuestro arsenal de producción musical. 
\end{itemize}


\subsection{La ganancia o gain y sus diferencias con el volumen}\label{sec:Gain}

Podríamos definir el volumen en el audio digital así: ``Cómo de ruidoso es algo \textbf{después} de salir por el amplificador de nuestro equipo de sonido.'' 

La ganancia sin embargo sería así: ``Cómo de ruidoso es algo \textbf{antes} de pasar por nuestros efectos y preamplificadores.''

La diferencia fundamental, además del tecnicismo de cuándo ocurre cada cosa, es que una variación en la ganacia produce una variación en el tono del sonido mientras que una variación en el volumen no produce cambios de tono.

\section{Efectos más usados o conocidos}



